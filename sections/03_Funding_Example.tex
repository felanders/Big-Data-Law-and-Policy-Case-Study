\section{Facebook Funding Example}
\label{sec:fac_fund}
\citet{abdalla_grey_2021} and \citet{oreskes_why_2019} argue that funding plays an important role in how research fields develop.
However, understanding Facebook's funding policy is not trivial.
First, there is little to no insight in what Facebook's internal researchers study, except for what the company chooses to publish.
Second, as we will see below, understanding what projects exactly are funded trough Facebook's research awards, is time-consuming and not always possible.
Third, finding out previous funding sources of some researchers is not always (easily) possible.

To gain at least an impression of Facebook's research funding practices, I decided to analyse one prominent Facebook research award in more depth:

After increasing public concerns about polarization on Facebook \citep{seetharaman_facebook_2020}, the company announced granting \$2 \gls{mn} to fund independent research on \textit{Misinformation and Polarization}\footnote{\citet{clegg_facebook_2020} presented an interesting spin on the issue in an article with the title \emph{Facebook Does Not Benefit from Hate}. There he explains that neither users nor advertisers want to see hate on Facebook and that \textquote{[t]here is no incentive for us to do anything but remove it.} With this he shifts attention away from the underlying problem (i.e., that Facebook's user-engagement above all attitude might foster hate-speech) towards the issue of content moderation}.
The request for proposals was published on February 24th 2020, on the research-awards sub-page of Facebook Research's website \citep{facebook_research_foundational_2020}. 
This website was later updated to present all the award recipients and their academic institutions.
Confusingly, the titles of proposals were not published on the research-awards sub-page, but only in the announcement of the award winners published on Facebook Research's Blog on August 7th 2020 \citep{leavitt_announcing_2020}.

In the context of the recent Cambridge-Analytica scandal, discussions about polarization and misinformation and the upcoming 2020 Presidential election \citep{isaac_facebook_2019}, the announcement of funding independent research on \textit{Misinformation and Polarization}, was certainly not only intended as showing good will and trying to understand where the problems on Facebook come from, but also served as a talking point in public. That is, to be able to say \emph{We do not only care about this issue, we also fund independent research about it}.

Now, research on \textit{Misinformation and Polarization} can mean many things and in the light of the strategies presented in the introduction (esp. distraction and manufacturing scientific dissent) one might ask: \emph{Did Facebook actually fund research addressing problems with its service, or was the funding targeted at creating distraction and talking points for upcoming debates?}

To understand this, and how transparent researchers are about the funding, I studied the web presence of the researchers that received funding through this award. 
I searched for their academic CV's and whether they disclosed the funds there, as well as if they reported about the funding on their web presence.
From eventual project descriptions, I tried to assess the relevance of the project for the context in which Facebook announced the grant  -- taking also the strategies mentioned above into account.

An overview of the results can be found in appendix \ref{sec:deep} and a more detailed version of the table is accessible in this \href{https://docs.google.com/spreadsheets/d/1WnfkCuypP09f67VIxDg-Vab5QsgU_te5OJUii6FIdtI/edit?usp=sharing}{Google spreadsheet}\footnote{The full url is: \url{https://docs.google.com/spreadsheets/d/1WnfkCuypP09f67VIxDg-Vab5QsgU_te5OJUii6FIdtI/edit?usp=sharing}}.\todo{update}

Almost one year after the announcement, we can observe that 15 out of 24 researchers did not disclose the Facebook funding on their web presence or CV.
One might argue that since the funding was disclosed on Facebook's page, there is no problem with this. Transparency was given.
While it is true, that the funding was disclosed somewhere on the internet, I want to argue, that does not necessarily mean transparency.

The problem is that many powerful and well-financed companies have an interest in the field of \gls{sns}, smartphone, or internet addiction. 
Examples are: Google, Apple, and Twitter, but also their Chinese Counterparts: Tencent, Baidu, and ByteDance, to only name the biggest players.
Researching all their pages, to find out about potential conflicts of interest, is a major effort.
Moreover, it does not stop at private companies.
As we know from Big Tobacco, but also from Facebook, funding research or \gls{pr} campaigns trough affiliated institutes or consultancies is nothing unusual \citep{oreskes_merchants_2010, frenkel_delay_2018}.

Therefore, even if technology companies disclosed all their official funding transparently, it would remain difficult -- that is time-consuming -- to find researchers' affiliations through Big Tech's sites.
Moreover, given that there might be less transparent funding sources and that a substantial percentage\footnote{While generalizing from the small sample in this example is not possible, it still gives an impression of how difficult understanding potential affiliations can be.} of researchers does not disclose all their funding (at least within one year), transparency is not given.

Another observation from this example is that there seems to be a mismatch between the funded projects and what Facebook publicly announced to be funding.
Unfortunately, for many research projects, the title was the only information available about the project content. 
This makes it difficult to assess whether the project substantively addresses the problem of misinformation and polarisation on Facebook.
Still, some of these projects leave the impression that the aim of their funding was to foster research in areas that might serve Facebook's public relation more than actually exploring the issues at hand (i.e., polarization and misinformation on Facebook).
Some examples raising this suspicion are:

The project on \emph{The contagion of misinformation} states: \textquote{The current attention to misinformation is still very focused on public social media such as Facebook and Twitter. 
There is however a growing body of evidence that private messaging services such as Telegram, WhatsApp or Link are just as likely to spread misinformation as publicly available social networks.} While WhatsApp (a Facebook company) is also mentioned as a potential platform for the spread of misinformation, it seems reasonable that such research would be cited by Facebook to blame Telegram for being a source of misinformation. 
That is an argument about the others also being bad, that is the strategy of distraction. 

Furthermore, it also seems reasonable that the project on \emph{Quantifying persistent effects of misinformation via neural signals} might produce a null or a weak finding. 
To me, it seems already difficult to define misinformation, not to speak about how to measure its persistent effect on the brain. 
If this project leads to a null or weak finding, Facebook could argue that misinformation does not harm the brain. That is a potential instance of distraction.

Lastly, the project on \emph{Micro-influencers as digital community health workers} is described to focus on \textquote{[determining] whether micro-influencers can be deployed to inform social media users about ticks and tick-borne disease using the principles of Inoculation Theory} \citep{cottingham_projects_nodate}.
A positive finding in this project could create a new line of defence for Facebook in denying the problem.
It could argue that people -- or organisations -- just have to \textit{engage} as micro influencers if they want to fight misinformation.

\paragraph{Limitations}
There are considerable limitations to this example:

First, the award was granted too recently, thus almost no research funded by this award was published yet. 
This limits much of the arguments above to being speculations.

Second, some scientists' web presences have not been updated since the funds were awarded on the 7th of August 2020.
While this further limits the information to be gained from this exemplary study, it still provides information on how easily researchers' industry funding can be understood, almost one year after they received a grant.