\section{Introduction}
\label{sec:intro}
%\addcontentsline{toc}{section}{Introduction}

\subsection{Motivation}
Over the past decades many different Industries have fought scientific results, scientists or even science in general, if the science produced results which where putting their profits at risk. Prime examples for this -- outlined among others in \citet{oreskes_merchants_2010} and \citet{cuveillier_forschung_2020} -- are: The Tobacco industry contesting the link between (secondhand-) smoking and cancer for decades, the Oil industry contesting the existence of anthropogenic (i.e. man made) climate change, or the Agro-Chemical industry diverting attention away from pesticides (esp. neonicotinoids\footnote{Neonicotinoids are chemicals used to protect plants from herbivore (plant eating) insects. The seeds of the plants are coated with these neonicotinoids which later migrate into all parts of the plant, which is why they are also called systemic insecticides. Since these chemicals migrate into all parts of the plant, they also end up in the pollen and this way come in contact with bees where they cause among others reduced foraging and reproduction \citep{whitehorn_neonicotinoid_2012}.}) towards other causes in the case of colony collapse disorder (bee death). 

These industries pursued different strategies to protect their interest, ranging from: \gls{pr} campaigns

\newpage

\begin{itemize}
    \item Watching a documentary on Arte \citep{cuveillier_forschung_2020} in a time where I was also attending the Seminar on Big, Data, Law, and Policy sparked the Idea that Big Tech might invoke similar strategies as Big Tobacco or Big Oil (i.e. seeding doubt \& diverting attention).
    \item In the mentioned documentary among others Naomi Oreskes was featured. In her Book "Merchants of Doubt" \citep{oreskes_merchants_2010} -- Co-authored with Erik M, Conawy -- she lays open some of the tactics that Big Tobacco and Big Oil use to manipulate public perception of scientific consensus.
    \item Reading up on this I found a recent paper which draws the connection between Big Tech's AI-Ethics funding and Big Tobaccos Strategies  \citep{abdalla_grey_2021}. This paper also references a hearing at the House Committee on Energy and Commerce, in which former Facebook executive Tim Kendall makes the connection to Big Tobacco stating: \textquote{We took a page from Big Tobacco’s playbook, working to make our offering addictive at the outset.} \citep{kendall_house_2020}.
    \item There is two more prominent former Facebook employees criticizing Facebook for their ethics and the overarching goal to optimize customer engagement: Sean Parker \citep{allen_sean_2017} and Tristan Harris \citep{metz_smartphones_2017}.
\end{itemize}

\subsection{Relevance for Big Data, Law, and Policy}
\begin{enumerate}
    \item Policy/Law-making is strongly influenced by public opinion.
    \item Especially for restricting regulations/laws solid scientific underpinning has become a very important prerequisite. If Big-tech is now artificially creating \textquote{scientific dissent}, policy-makers leeway for introducing regulation becomes much more restricted.
    \item Thus -- because of the potential manipulation by Big Tech and the resulting policy impacts -- understanding the modus-operandi of Big Tech is important to inform policy makers and understand the debate.
\end{enumerate}

In this case study I will focus on Facebook -- as one important player of Big Tech --, the issue of social media addiction and the science of this topic.
A reason for me to choose this topic is that while AI-Ethics is very prominent at the moment, it is an inherently ambiguous and normative topic, meaning that there will never be a scientific consensus. 

Furthermore, the issue about AI-Ethics funding is being discussed (at least to some extent) pubblicly e.g. in \citep{abdalla_grey_2021} but also in reports about the Facebook-funded AI-Ethics chair at TU-Munich.

I think that the basic metric Facebook is maximizing for i.e. their main goal and driver: \textquote{user-engagement} is an important object of study since it constitutes the basis for most of Facebook's decisions, and through this also a considerable amount of time for about 2bn people.
