\section{Introduction}
\label{sec:intro}
%\addcontentsline{toc}{section}{Introduction}

\subsection{Motivation}
Over the past decades many different industries have fought scientific results, scientists or even science in general, if the scientific results where putting their profits at risk. 
Some prime examples -- outlined among others in \citet{oreskes_merchants_2010} and \citet{cuveillier_forschung_2020} -- are: 
Big Tobacco (i.e. the largest tobacco companies) contesting the link between (secondhand-) smoking and cancer for decades, 
Big Oil (i.e.the largest oil companies) contesting the existence of anthropogenic (i.e. man made) climate change, 
or the agro-chemical industry diverting attention away from pesticides (esp. neonicotinoids\footnote{Neonicotinoids are chemicals used to protect plants from herbivore (plant eating) insects. The seeds of the plants are coated with these neonicotinoids which later migrate into all parts of the plant, which is why they are also called systemic insecticides. Since these chemicals migrate into all parts of the plant, they also end up in the pollen and this way come in contact with bees where they cause among others reduced foraging and reproduction \citep{whitehorn_neonicotinoid_2012}.}) towards other causes in the case of colony collapse disorder (i.e. bee death).


These reports of industries -- ruled by few large players -- actively fighting scientific evidence over the last couple of decades, raises the question:
\begin{quote}
\itshape
Does Big Tech engage in similar activities?
\end{quote}

As pointed out in \citep{abdalla_grey_2021}, defining Big Tech is not straight-forward, for the scope of this paper, however, a strict definition will not be necessary.
The \textquote{Big Five} digital technology firms namely: Alphabet (the parent company of Google), Amazon, Apple, Facebook, and Microsoft can serve as a first starting point while keeping in mind that there are many other large and powerful technology companies.

The Big Five alone, have recently\footnote{As of July 1st 2021.} been ranked in positions 1, 2, 4, 5, and 6 in in terms of worldwide market capitalization (i.e. the number of shares times the current market price).
The only company surpassing some of Big Tech in this ranking was the Saudi Arabian oil company Saudi Amarco \citep{noauthor_largest_2021}. 
Furthermore, all five companies together amount to a market capitalization of \$ 8693 \gls{tn} as compared to as compared to the GDP of the USA which was estimated as \$22.06 \gls{tn} in the first Quarter of 2021 \citep{noauthor_gross_2021}.

Not only are these companies large and powerful, they also shape the development of this strongly human facing technology. That is they have a large influence in defining the technologies most humans are using on a daily basis. 
In 2020 more than 80\% of the U.S. population -- 18 years and older -- owned a smartphone and used it at least once per month, of them 46\% reported to use their smartphone between 5 and 6 hours daily, with a general daily average of 3 hours and 6 minutes \citep{noauthor_us_2021, noauthor_smartphone_2021, noauthor_time_2020}. 

Furthermore, a recent paper addressed the issue of corporate research funding by focusing on Big Tech's\footnote{The authors define Big Tech as: \textquote{Google, Amazon, Facebook, Microsoft, Apple, Nvidia, Intel, IBM, Huawei, Samsung, Uber, Alibaba, Element AI, OpenAI} \citep{abdalla_grey_2021}.} funding efforts in \gls{ai}-ethics research \citep{abdalla_grey_2021}.
The authors studied the influence Big Tech might have on a seemingly \textquote{scientific} definition of what is deemed to be ethical \gls{ai} and what is not\footnote{Leaving aside whether there is such a thing as a scientific definition of ethicality.} and compared some of the strategies with the tactics of Big Tobacco.

\subsection{Focus}
In this case study I will focus on Facebook and more specifically on the issue of social media addiction or problematic use of \gls{sns}.
Facebook is the 6th largest company in the world (in terms of market capitalization), and reports about 2.6 \gls{bn} daily and 3.3 \gls{bn} monthly active people across all of their services (Facebook, WhatsApp, Messenger and Instagram) in 2020.
They also report a yearly average revenue per person of \$ 27.51 \todoc{cite}.

Facebook reported spending 21\% of its revenue that is \$18'447 \gls{mn} on research and development in 2020, while it spent 13\% (\$ 11'591 \gls{mn}) of its revenue on Marketing and Sales, and reported 8\% (\$ 6,564 \gls{mn}) of its revenue to be  general and administrative costs. \todoc{cite}
To set the spending on research and development in context, all German higher education institutions together spent €61.01 \gls{bn} while it earned €32.83 \gls{bn} in 2019, leaving expenses of €28.18 \gls{bn} \footnote{about \$ 31.55 \gls{bn} at a exchange rate of 1.11957 as of \todoc{cite eidgenössische} } \todoc{cite ausgaben (und einnahmen) von hochschulen...}. The state of California spent \$41.48 \gls{bn} on its education system in 2017. \todoc{cite higher education spending} 

Furthermore, Facebook was reported to have spent around \$19.6 \gls{mn} on lobbying in 2020\todoc{cite opensecrets and senate}.
For comparison in 2020 Amazon spent about \$ 18.7 \gls{mn}, Google \$ 8.85 \gls{mn}, Exxon Mobile (oil) \$ 8.69 \gls{mn}, and Philipp Morris (tobacco) \$ 6.95 \gls{mn}. \todoc{cite senate and business insider maybe also WSJ}
Finally, Facebook also spent about \$ 566 thousand on election campaigning through the Facebook INC PAC \todoc{cite open secrets}.




The focus on Facebook is 
This topic is interesting because a big part of the 
 -- as one important player of Big Tech --, the issue of social media addiction and the science of this topic.
A reason for me to choose this topic is that while AI-Ethics is very prominent at the moment, it is an inherently ambiguous and normative topic, meaning that there will never be a scientific consensus. 

Furthermore, the issue about AI-Ethics funding is being discussed (at least to some extent) pubblicly e.g. in \citep{abdalla_grey_2021} but also in reports about the Facebook-funded AI-Ethics chair at TU-Munich.

I think that the basic metric Facebook is maximizing for i.e. their main goal and driver: \textquote{user-engagement} is an important object of study since it constitutes the basis for most of Facebook's decisions, and through this also a considerable amount of time for about 2bn people.


\subsection{Relevance for Law and Policy}


\subsection{Strategies}
Oreskes \& Conaway present different cases and outline a whole toolkit of strategies to attack science:

These industries pursued different strategies to protect their interest, ranging from:
\begin{itemize}
\item \gls{pr}
\item Denial and publishing in non-peer reviewed "self"-editored journals (creating the apparence of science)
\item Creating a cadre of experts to be used in the future. also subsidizing/funding dissent.
\item using the balanced reporting imperative of us media
\item Diverting research attention (also birds can cause lung cancer) 
\end{itemize}


Also \cite{abdalla_grey_2021} looked at the tools ...

It is important to note that while Merchants of Doubt lays out these techniques very clearly, they are not as easy to detect as it may seem. Naomi Oreskes herself reports \begin{quote}
....
\end{quote} \citep{oreskes_merchants_2010} 

\newpage

\begin{itemize}
    \item Watching a documentary on Arte \citep{cuveillier_forschung_2020} in a time where I was also attending the Seminar on Big, Data, Law, and Policy sparked the Idea that Big Tech might invoke similar strategies as Big Tobacco or Big Oil (i.e. seeding doubt \& diverting attention).
    \item In the mentioned documentary among others Naomi Oreskes was featured. In her Book "Merchants of Doubt" \citep{oreskes_merchants_2010} -- Co-authored with Erik M, Conawy -- she lays open some of the tactics that Big Tobacco and Big Oil use to manipulate public perception of scientific consensus.
    \item Reading up on this I found a recent paper which draws the connection between Big Tech's AI-Ethics funding and Big Tobaccos Strategies  \citep{abdalla_grey_2021}. This paper also references a hearing at the House Committee on Energy and Commerce, in which former Facebook executive Tim Kendall makes the connection to Big Tobacco stating: \textquote{We took a page from Big Tobacco’s playbook, working to make our offering addictive at the outset.} \citep{kendall_house_2020}.
    \item There is two more prominent former Facebook employees criticizing Facebook for their ethics and the overarching goal to optimize customer engagement: Sean Parker \citep{allen_sean_2017} and Tristan Harris \citep{metz_smartphones_2017}.
\end{itemize}

\subsection{Relevance for Big Data, Law, and Policy}
\begin{enumerate}
    \item Policy/Law-making is strongly influenced by public opinion.
    \item Especially for restricting regulations/laws solid scientific underpinning has become a very important prerequisite. If Big-tech is now artificially creating \textquote{scientific dissent}, policy-makers leeway for introducing regulation becomes much more restricted.
    \item Thus -- because of the potential manipulation by Big Tech and the resulting policy impacts -- understanding the modus-operandi of Big Tech is important to inform policy makers and understand the debate.
\end{enumerate}


