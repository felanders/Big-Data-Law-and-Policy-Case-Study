\section{Introduction}
\label{sec:intro}
%\addcontentsline{toc}{section}{Introduction}

\subsection{Motivation}
Over the past decades many different industries have fought scientific results, scientists or even science in general, if the scientific results where putting their profits at risk. 
Some prime examples -- outlined among others in \citet{oreskes_merchants_2010} and \citet{cuveillier_forschung_2020} -- are: 
Big Tobacco (i.e. the largest tobacco companies) contesting the link between (secondhand-) smoking and cancer for decades, 
Big Oil (i.e.the largest oil companies) contesting the existence of anthropogenic (i.e. man made) climate change, 
or the agro-chemical industry diverting attention away from pesticides (esp. neonicotinoids\footnote{Neonicotinoids are chemicals used to protect plants from herbivore (plant eating) insects. The seeds of the plants are coated with these neonicotinoids which later migrate into all parts of the plant, which is why they are also called systemic insecticides. Since these chemicals migrate into all parts of the plant, they also end up in the pollen and this way come in contact with bees where they cause among others reduced foraging and reproduction \citep{whitehorn_neonicotinoid_2012}.}) towards other causes in the case of colony collapse disorder (i.e. bee death).


These reports of industries -- ruled by few large players -- actively fighting scientific evidence over the last couple of decades, raises the question:
\begin{quote}
\itshape
Does Big Tech engage in similar activities?
\end{quote}

As pointed out in \citep{abdalla_grey_2021}, defining Big Tech is not straight-forward, for the scope of this paper, however, a strict definition will not be necessary.
The \textquote{Big Five} digital technology firms namely: Alphabet (the parent company of Google), Amazon, Apple, Facebook, and Microsoft can serve as a first starting point while keeping in mind that there are many other large and powerful technology companies.

The Big Five alone, have recently\footnote{As of July 1st 2021.} been ranked in positions 1, 2, 4, 5, and 6 in in terms of worldwide market capitalization (i.e. the number of shares times the current market price).
The only company surpassing some of Big Tech in this ranking was the Saudi Arabian oil company Saudi Amarco \citep{noauthor_largest_2021}. 
Furthermore, all five companies together amount to a market capitalization of about \$8.693 \gls{tn} which is about 40\% of the current U.S. gross domestic product which -- in the first Quarter of 2021 -- was estimated to amount to \$22.06 \gls{tn} \citep{bea_gross_2021}.

Not only are these companies large and powerful, they also shape the development of this strongly human facing technology. That is they have a large influence in defining the technologies most humans are using on a daily basis. 
In 2020 more than 80\% of the U.S. population -- 18 years and older -- owned a smartphone and used it at least once per month, of them 46\% reported to use their smartphone between 5 and 6 hours daily, with a general daily average of 3 hours and 6 minutes \citep{odea_us_2021, odea_smartphone_2021, statista_research_department_time_2020}. 

Furthermore, a recent paper addressed the issue of corporate research funding by focusing on Big Tech's\footnote{The authors define Big Tech as: \textquote{Google, Amazon, Facebook, Microsoft, Apple, Nvidia, Intel, IBM, Huawei, Samsung, Uber, Alibaba, Element AI, OpenAI} \citep{abdalla_grey_2021}.} funding efforts in \gls{ai}-ethics research \citep{abdalla_grey_2021}.
The authors studied the influence Big Tech might have on a seemingly \textquote{scientific} definition of what is deemed to be ethical \gls{ai} and what is not\footnote{Leaving aside whether there is such a thing as a scientific definition of ethics.} and compared some of the strategies with the tactics of Big Tobacco.

\subsection{Focus}
In this case study I will focus on Facebook and more specifically on the issue of social media addiction or problematic use of \gls{sns}.
Facebook is the 6th largest company in the world (in terms of market capitalization), and reports estimates of 2.6 \gls{bn} daily and 3.3 \gls{bn} monthly active people across all of their services (Facebook, WhatsApp, Messenger and Instagram) in 2020.
They also report a yearly average revenue per person of \$27.51 \citep{facebook_inc_annual_2021}.

Furthermore, Facebook reported spending 21\% of its revenue that is \$18.45 \gls{bn} on research and development in 2020, while it spent 13\% (\$11.59 \gls{bn}) of its revenue on Marketing and Sales, and reported 8\% (\$6.56 \gls{bn}) of its revenue to be  general and administrative costs. \citep{facebook_inc_annual_2021}
To set the spending on research and development in context, all German higher education institutions together and \textit{across all disciplines} spent €61.01 \gls{bn} while it earned €32.83 \gls{bn} in 2019, leaving expenses of €28.18 \gls{bn}\footnote{about \$31.55 \gls{bn} at a mean exchange rate of 1.11957 for 2019 \citep{estv_jahresmittelkurse_2021} .} \citep{statistisches_bundesamt_ausgaben_2021, statistisches_bundesamt_einnahmen_2021}. The state of California spent \$41.48 \gls{bn} on its education system in 2017 \citep{duffin_us_2020}.

These numbers indicate Big Tech's agenda-setting power for research in  information technology.
Certainly it is not per se (i.e. inherently) bad that Big Tech is spending large sums on research and development, also because a fair share of these expenses will most probably be devoted to product development, that is, actually building the technology and services these companies sell.

In recent years there has been a public debate around Big Tech and \gls{ai}-ethics.
Examples for this are: The public debate about the firings of Google \gls{ai} ethics researchers Timnit Gebru and Margaret Mitchell \citep{timnit_gebru_i_2020, simonite_prominent_2020, agency_google_2021, noauthor_margaret_2021}; 
The controversy around the Facebook funded \gls{ai}-ethics Institute at TU-Munich \citep{kover_warum_2019, kreiss_vielsagender_2019, kreye_facebook_2019, hauck_facebook_2019, thiel_kommentar_2019};
Reports on gender discrimination of Apple's AI powered credit card \citep{heinemeier_hansson_dhh_2019, vigdor_apple_2019, hegemann_apple_2019, mahdawi_apples_2019};
The discussions around Facebook's role in elections, misinformation and polarization \citep{seetharaman_facebook_2020, kates_facebook_2017, klein_what_2020, boxell_cross-country_2020, newton_how_2020, rosen_smart_2019}.

While, these issues focus mostly on the impact of \gls{ai} and machine learning systems and how these should be regulated, they only slightly touch on issues resulting from Google and Facebook's effort to maximize: \emph{user engagement}. 
User engagement is  Google and Facebook's way to call the amount of attention or time users (i.e. humans) spend on a digital service.
Some of this attention is then monetized by selling advertisement-space on the respective service, as has been described recently by some prominent former Google and Facebook employees:
\begin{itemize}
    \item Roger McNamee, who was an early investor of Google and Facebook and an advisor to Facebook's team, describes Google and Facebook as \textquote{Borrowing techniques from the gambling industry, [...]  exploit[ing] human nature, creating addictive behaviors, that compel consumers to check for new messages, respond to notifications, and seek validation from technologies whose only goal is to generate profits for their owners.} \citep{mcnamee_i_2017}.
    \item Tristan Harris, a former product manager an design ethicist at Google, started a non profit organization called Time Well Spent, which later developed into the \href{https://www.humanetech.com/who-we-are}{Center for Humane Technology} that aims at holding the tech industry accountable for the ways they try to nudge users to spend more time on their services. \citep{metz_smartphones_2017}
    \item There is also Sean Parker co-founder of the file sharing platform napster and founding director of Facebook. In an interview he reported: \textquote{[...] Facebook being the first of them, ... was all about: 'How do we consume as much of your time and conscious attention as possible?'} \textquote{And that means that we need to sort of give you a little dopamine hit every once in a while [...]}, \textquote{[...] exactly the kind of thing that a hacker like myself would come up with, because you're exploiting a vulnerability in human psychology.} \citep{allen_sean_2017}
    \item Finally, former Facebook executive Tim Kendall made a direct connection to Big Tobacco's tactics stating: \textquote{We took a page from Big Tobacco’s playbook, working to make our offering addictive at the outset.} \citep{kendall_house_2020} during a testimony before the U.S. House committee on energy and commerce.
\end{itemize}


\subsection{Relevance for Law and Policy}
\textit{How does this issue relate to law and policy?}

First, public opinion can be a powerful force in the process of legislation and policy making. 
Naomi Oreskes and Erik M. Conaway for example report in their book "Merchants of Doubt" that the Reagan-administration interfered in the the publication of a peer review committee report scrutinizing the scientific evidence on the harms of acid rain.
The administration delayed the publication and weakened the executive summary (without informing all members of the committee) in order not to give the opposition: "the ammunition [they] needed to push acid rain controls trough Congress", as Congressman Norman D'Amours was quoted saying \citep[p. 95 ff.]{oreskes_merchants_2010}.

Second, Judicial Rulings are recently also influenced by scientific findings and expert opinions. Big Tobacco for example relied on a handful of experts which -- in line with the industry -- claimed that there was scientific consensus about the causal link between smoking and cancer. 
The industry even pushed the careers of \textquote{promising} researchers trough funding in order to create a cadre of benevolent experts and at the same time creating artificial scientific dispute \citep{oreskes_merchants_2010}.

Third, in 2019 U.S. Senator Josh Hawley (R-MO) introduced the Social Media Addiction Reduction Technology (SMART) Act, a bill regulating addictive features (such as: infinite scrolling feeds and  video auto-play) in websites and social media. 
Furthermore, the bill would have introduced a default daily limit of 30 minutes per service across devices, with users having to change this limit on a monthly basis \citep{hawley_social_2019}.

Opponents of the proposal argued that there was no scientific consensus about social media addiction and  Michael Beckerman (CEO of the Internet Association, "a trade group including Google, Facebook, Amazon, Twitter, [...], and Snapchat"\citep{rifkin_social_2019}) was quoted having said: \textquote{Policy proposals must be evidence-based.} \citep{stewart_josh_2019} perpetuating the strategy of Big Tobacco and Big Oil to demand scientific consensus while at the same time doubting the consensus.

After receiving fairly broad public attention \citep{rosen_smart_2019, chen_new_2019, clukey_lawmaker_2019} the bill did not receive a vote and died \citep{govtrackus_smart_nodate}.

Finally, Facebook not only has a very large budget for research and development, and a large user base covering a  substantial portion of the global, Facebook also reported spending around \$19.6 \gls{mn}  on lobbying the U.S. Senate and House in 2020 alone \citep{opensecrets_facebook_2021-1}\footnote{The official filings can be found in: \citep{maurer_ld-2_2020-2, maurer_ld-2_2020-1, maurer_ld-2_2020, maurer_ld-2_2021}}.
For comparison in 2020 Amazon spent about \$18.7 \gls{mn}, Google \$8.85 \gls{mn}, Exxon Mobile (oil) \$8.69 \gls{mn}, and Philipp Morris (tobacco) \$6.95 \gls{mn} \citep{khaled_facebook_2021, opensecrets_amazoncom_2021, opensecrets_alphabet_2021, opensecrets_exxon_2021, opensecrets_philip_2021}.
Furthermore, Facebook spent about \$566 thousand on election campaigning through the Facebook INC. PAC. \citep{opensecrets_facebook_2021}.

Summarizing, public opinion can be influenced by scientific findings and also have a strong influence on legislators. Furthermore, Facebook, Google and others have very large research and lobbying budgets, through which they can influence the research and policy agenda. 

Moreover, historical findings about the strategies of Big Tobacco and others to protect their extractive profit optimizing practices reveal analogies to Big Tech.
Therefore, understanding the debate about \gls{sns} addiction or problematic use as well as the way Facebook and other tech firms fund research and push certain findings  will help legislators and judiciaries to make informed decisions.


\subsection{Strategies}
There are various strategies we now know of which Big Tobacco and Big Oil, the Agrochemical Industry, and others used to fight uncomfortable scientific findings\footnote{Many of these are outlined in the Book \textquote{Merchants of Doubt} \citep{oreskes_merchants_2010}, moreover, \citep{abdalla_grey_2021} compare Big Tobaccos playbook with how Big Tech is influencing \gls{ai}-ethics research and  finally a story recently published on Unearthed reveals Exxon Mobile's lobbying strategies \todoc{cite}}.
Knowing and understanding these strategies and argumentation structures will help to identify them, which is why I will shortly outline them below:


refer to strategies employed by Big Tobacco and compare Big Techs Funding behaviour with th
Oreskes \& Conaway present different cases and outline a whole toolkit of strategies to attack science:

These industries pursued different strategies to protect their interest, ranging from:
\begin{itemize}
\item \gls{pr}
\item Denial and publishing in non-peer reviewed "self"-editored journals (creating the apparence of science)
\item Creating a cadre of experts to be used in the future. also subsidizing/funding dissent.
\item using the balanced reporting imperative of us media
\item Diverting research attention (also birds can cause lung cancer) 
\end{itemize}


Also \cite{abdalla_grey_2021} looked at the tools ...

It is important to note that while Merchants of Doubt lays out these techniques very clearly, they are not as easy to detect as it may seem. 
After they explain how a few scientists (among them Edward Krug) created the impression of a scientific debate around Acid Rain, while in essence there was none, they write \begin{quote}
And while we are embarrassed to admit it, in the early 1990s one of us (N.O.) used Krug's arguments in an introductory earth science class at Dartmouth College to teach \textquote{both sides} of the acid rain \textquote{debate}
\end{quote} \citep[p. 103]{oreskes_merchants_2010} 

\newpage