\section{Introduction}
\label{sec:intro}
%\addcontentsline{toc}{section}{Introduction}

\subsection{Motivation}
Over the past decades various industries have fought scientific results, scientists, or even science in general, if the scientific results were putting their profits at risk. 
Some prime examples -- outlined among others in \citet{oreskes_merchants_2010} and \citet{cuveillier_forschung_2020} -- are: 
Big Tobacco (i.e., the world's largest tobacco companies) contesting the link between (second-hand) smoking and cancer for decades, 
Big Oil (i.e., the largest oil companies) contesting the existence of anthropogenic (man made) climate change, 
or the agrochemical industry diverting attention away from pesticides (esp. neonicotinoids\footnote{Neonicotinoids are chemicals used to protect plants from herbivore (plant eating) insects. The seeds of these plants are coated with neonicotinoids which later migrate into all parts of the plant, which is why they are also called \textit{systemic} insecticides. Since these chemicals migrate into all parts of the plant, they also end up in the pollen and reach bees this way, where they cause among others reduced foraging and reproduction \citep{whitehorn_neonicotinoid_2012}.}) towards other causes, in the issue of colony collapse disorder (i.e., bee death).

These reports of industries -- ruled by few large players -- that actively fought scientific evidence over the last couple of decades, raises the question:
\begin{quote}
\itshape
Does Big Tech engage in similar activities?
\end{quote}

As pointed out in \citep{abdalla_grey_2021}, defining Big Tech is not trivial.
For the scope of this paper, however, a precise definition will not be necessary.
The \textquote{Big Five} digital technology firms namely: Alphabet (the parent company of Google), Amazon, Apple, Facebook, and Microsoft can serve as a mental starting point, while remembering that there are many other large and powerful technology companies\footnote{\citet[p 2]{abdalla_grey_2021} define Big Tech as: \textquote{Google, Amazon, Facebook, Microsoft, Apple, Nvidia, Intel, IBM, Huawei, Samsung, Uber, Alibaba, Element AI, OpenAI}.\label{foot:big_tech}}.

Why start with the Big Five? They, have recently\footnote{As of \today.}been ranked in positions 1, 2, 4, 5, and 6 in terms of worldwide market capitalisation (i.e., the number of shares times the current market price).
The only firm surpassing some of the Big Five companies in this ranking, is Saudi Amarco the Saudi Arabian oil company \citep{noauthor_largest_2021}. 
Furthermore, all five companies together amount to a market capitalisation of about \$9.13 \gls{tn} which is about 41\% of the current US gross domestic product, which -- in the first Quarter of 2021 -- was estimated to amount to \$22.06 \gls{tn} \citep{bea_gross_2021}.

Not only are these companies large and powerful, they also predominantly shape the development of large parts of human facing digital technology. 
That is, they control -- in large part -- the design of technologies most humans are using on a daily basis. 
In 2020 more than 80\% of the US population -- 18 years and older -- owned a smartphone and used it at least once per month, 46\% of them reported using their smartphone between 5 and 6 hours daily, with an overall daily average of 3 hours and 6 minutes \citep{statista_research_department_time_2020, odea_us_2021, odea_smartphone_2021}. 

Finally, the idea that Big Tech might copy some of Big Tobacco's strategies has been raised already in the scientific literature. 
\citet{abdalla_grey_2021} addressed the issue of corporate research funding by focusing on Big Tech's\footref{foot:big_tech} funding efforts in \gls{ai}-ethics research.
The authors studied the influence Big Tech might have on a seemingly \textquote{scientific} definition of what is deemed to be ethical \gls{ai} and what is not\footnote{Let us leave aside the question of whether there is such a thing as a scientific definition of what is ethical.}. 
In their paper, they also compared some strategies with the tactics of Big Tobacco (as a paragon for manipulation in the sciences).

\subsection{Focus}
This case study will focus on Facebook and more specifically on the issue of \gls{sns} addiction or problematic \gls{sns} use.

Facebook has  -- since its funding in 2004 -- become one of the most powerful private companies in the world. 
It is the company with the 6th largest market capitalisation (\$976 \gls{bn}\citep{noauthor_largest_2021} as of today), and reported estimates of 2.6 \gls{bn} daily and 3.3 \gls{bn} monthly active people across all of their services (Facebook, WhatsApp, Messenger and Instagram) in 2020.
For all these users, Facebook reported an average yearly revenue of \$27.51 \citep{facebook_annual_2021}.

From these revenues Facebook reported spending 21\%, that is, \$18.45 \gls{bn} on research and development in 2020, while it spent 13\% (\$11.59 \gls{bn}) on Marketing and Sales, and reported 8\% (\$6.56 \gls{bn}) to be general and administrative costs \citep{facebook_annual_2021}.
To set the spending on research and development in context, all German higher education institutions together and \textit{across all disciplines} spent €61.01 \gls{bn} while they earned €32.83 \gls{bn} in 2019, leaving expenses of €28.18 \gls{bn} (about \$31.55 \gls{bn}\footnote{At a mean exchange rate of 1.11957 for 2019 \citep{estv_jahresmittelkurse_2021} .}) \citep{statistisches_bundesamt_ausgaben_2021, statistisches_bundesamt_einnahmen_2021}. 
Moreover, the state of California reported direct general expenditures of \$41.48 \gls{bn} for all its higher education in 2017 \citep{duffin_us_2020}.

Certainly, it is not inherently bad that Big Tech\footnote{Alphabet e.g., reports \$182.5 \gls{bn} revenues and \$27.6 \gls{bn} spending on research and development \citep{alphabet_inc_annual_2021}.} is spending large sums on research and development, esp. because a large part of these expenses is -- most probably -- devoted to product development. 
That is, building the technology and services these companies sell.
Nevertheless, these numbers give a sense of Big Tech's agenda-setting power in information technology research.

Recently, there has been a public debate around Big Tech' ethics, often connected to \gls{ai}-ethics.
Some examples are: The public debate about the firings of Google \gls{ai}-ethics researchers Timnit Gebru and Margaret Mitchell \citep{timnit_gebru_i_2020, simonite_prominent_2020, agency_google_2021, noauthor_margaret_2021}; 
The controversy around the Facebook funded \gls{ai}-ethics institute at TU Munich \citep{kover_warum_2019, kreiss_vielsagender_2019, kreye_facebook_2019, hauck_facebook_2019, thiel_kommentar_2019};
Reports on gender discrimination of Apple's AI-powered credit card \citep{heinemeier_hansson_dhh_2019, vigdor_apple_2019, hegemann_apple_2019, mahdawi_apples_2019};
The discussions about Facebook's role in elections, misinformation, and polarisation \citep{kates_facebook_2017, rosen_smart_2019, klein_what_2020, boxell_cross-country_2020, newton_how_2020, seetharaman_facebook_2020}.

While, these issues focus predominantly on the impact of \gls{ai} and machine learning systems and how these should be regulated, they only slightly touch on issues resulting from Google and Facebook's effort to maximise \emph{user engagement}. 
User engagement is  Google and Facebook's term for the amount of attention or time users (i.e., humans) spend on a digital service.
This attention is then monetised by selling advertisements on the respective service\footnote{In a hearing before the US Senate committee on the judiciary, Mark Zuckerberg (CEO of Facebook), famously responded, to the question of how Facebook makes money \textquote{Senator, we run ads.} \citep{ noauthor_senator_2018, noauthor_facebook_2018}}. 
The ramifications of this business-model have been described recently by some prominent former Google and Facebook employees:
\begin{itemize}
    \item \textit{Roger McNamee}, who was an early investor of Google and Facebook and an early advisor to Facebook's leadership team, describes Google and Facebook as \textquote{Borrowing techniques from the gambling industry, [...]  exploit[ing] human nature, creating addictive behaviors, that compel consumers to check for new messages, respond to notifications, and seek validation from technologies whose only goal is to generate profits for their owners.} \citep{mcnamee_i_2017}.
    \item \textit{Tristan Harris}, a former product manager and design ethicist at Google, who left the company to start a non-profit organisation called Time Well Spent. Later, this developed into the \href{https://www.humanetech.com/who-we-are}{Center for Humane Technology}, which aims at holding the tech industry accountable for how they try to nudge users into spending more time on their services. \citep{metz_smartphones_2017}
    \item \textit{Sean Parker,} co-founder of the file sharing platform Napster and founding director of Facebook, reported in an interview: \textquote{[...] Facebook being the first of them, ... was all about: 'How do we consume as much of your time and conscious attention as possible?'} \textquote{And that means that we need to sort of give you a little dopamine hit every once in a while [...]}, \textquote{[...] exactly the kind of thing that a hacker like myself would come up with, because you're exploiting a vulnerability in human psychology.} \citep{allen_sean_2017}
    \item Finally, former Facebook executive \textit{Tim Kendall} made a direct connection to Big Tobacco's tactics when stating: \textquote{We took a page from Big Tobacco’s playbook, working to make our offering addictive at the outset.} \citep{kendall_house_2020} during a testimony before the US House committee on energy and commerce.
\end{itemize}


\subsection{Relevance for Law and Policy}
\textit{How does this issue relate to law and policy?}

First, public opinion can be a powerful force in the process of legislation and policymaking. 
Naomi Oreskes and Erik M. Conaway for example, report in their book \emph{Merchants of Doubt} that the Reagan administration interfered in the publication of a peer review committee's report scrutinising the scientific evidence on the harms of acid rain.
The administration delayed the publication and weakened the executive summary (without informing all members of the committee) in order not to give the opposition \textquote{[...] the ammunition [they] needed to push acid rain controls trough Congress}, as congressman Norman D'Amours was quoted saying \citep[p 95 ff]{oreskes_merchants_2010}.

Second, also judicial rulings are influenced by scientific findings and expert opinions. 
Big Tobacco for example relied on a handful of experts which -- in line with industry interests -- claimed that there was no scientific consensus about the causal link between smoking and cancer. 
The industry, furthermore, pushed the careers of \textquote{promising} researchers trough generous funding.
This way they created a cadre of benevolent experts with scientific credentials while at the same time artificially creating scientific dissent \citep{oreskes_merchants_2010}.

Third, \gls{sns} addiction has already been -- and the power of Big Tech currently is\footnote{In April 2021 US Senator Josh Hawley (R-MO) introduced a bill to \textquote{Bust UP Big Tech} \citep{govtrackus_bust_2021}.} -- a topic for legislators.
In 2019 US Senator Josh Hawley (R-MO) introduced the Social Media Addiction Reduction Technology (SMART) Act, a bill regulating addictive features (such as infinite scrolling feeds or video auto-play) of websites and \gls{sns}.
Furthermore, the bill would have introduced a monthly renewing default daily time limit of 30 minutes per service across devices. That is, users would have had to change this limit on a monthly basis if they wanted to use services longer than 30 minutes daily \citep{hawley_social_2019}.
However, after receiving fairly broad public attention \citep{rosen_smart_2019, chen_new_2019, clukey_lawmaker_2019}, the bill did not receive a vote and died \citep{govtrackus_smart_nodate}.

Opponents of the proposal argued that there was no scientific consensus about social media addiction.
For example, Michael Beckerman (CEO of the Internet Association, \textquote{a trade group including Google, Facebook, Amazon, Twitter, [...], and Snapchat}\citep{rifkin_social_2019}) was quoted having said: \textquote{Policy proposals must be evidence-based.} \citep{stewart_josh_2019}. 
By this, he perpetuated the strategy of Big Tobacco and Big Oil to demand scientific consensus while at the same time seeding doubt about the consensus.

Finally, Facebook not only has a large budget for research and development, and a large user base -- covering a substantial portion of the global population -- Facebook also reported spending around \$19.6 \gls{mn} on lobbying the US Congress in 2020 alone \citep{opensecrets_facebook_2021-1}\footnote{The official filings can be found in: \citep{maurer_ld-2_2020-2, maurer_ld-2_2020-1, maurer_ld-2_2020, maurer_ld-2_2021}}.
For comparison, in 2020 Amazon spent about \$18.7 \gls{mn}, Google \$8.85 \gls{mn}, Exxon Mobile (oil) \$8.69 \gls{mn}, and Philipp Morris (tobacco) \$6.95 \gls{mn} \citep{khaled_facebook_2021, opensecrets_amazoncom_2021, opensecrets_alphabet_2021, opensecrets_exxon_2021, opensecrets_philip_2021}.
Furthermore, Facebook spent about \$566 thousand on election campaigning through the Facebook Inc. PAC \citep{opensecrets_facebook_2021}.

In Summary, public opinion can be influenced by scientific findings and also have a strong influence on legislators. Furthermore, Facebook, Google, and others have very large research and lobbying budgets, through which they influence the research and policy agendas.
Moreover, findings about which strategies Big Tobacco and others used to protect their extractive, profit-maximising practices show analogies to Big Tech.

Therefore, understanding the debate about \gls{sns} addiction or problematic use, as well as the way Facebook and other tech firms fund research and promote certain findings,  will help legislators and judiciaries to make better informed decisions.

\subsection{Strategies}
There are various strategies we know of which Big Tobacco, Big Oil, the Agrochemical Industry, and others used to fight profit threatening or inconvenient scientific findings.
Knowing and understanding these tactics and argumentation structures will help to identify them in debates.
The most important strategies are\footnote{The book \emph{Merchants of Doubt} by \citet{oreskes_merchants_2010} served as a source for these. Moreover, \citet{abdalla_grey_2021} compare Big Tobaccos' playbook with how Big Tech is influencing \gls{ai}-ethics research. Finally, a story recently published on \href{https://unearthed.greenpeace.org/}{Unearthed} -- Greenpeace UK's investigative journalism project -- revealed Exxon Mobile's lobbying strategies \citep{carter_inside_2021}.}:

\paragraph{1 Denial} (The science is not conclusive): This is an overarching argument and aim of many of the following tactics. 
That is, to claim that there is (reasonable) doubt about the scientific findings, which would justify regulation -- for example a CO\textsubscript{2} cap and trade scheme or a smoking ban. 
This tactic can invoke points \XeTeXglyph244 \,2, 3, 4 and 5 in this list.

\paragraph{2 Distraction} (Other things are also bad):  An example for this is Big Tobacco, that funded research about other causes of cancer, such as \textquote{stress, genetic inheritance, and the like} \citep[p 14]{oreskes_merchants_2010}. 
Among others, they claimed that since asbestos was a likely cause of cancer, one could never be certain if smoking had caused a certain instance of lung cancer, or if the cause was exposure to asbestos. 
That is, they used the fact that the science on lung cancer was based on statistical relationships and thus necessarily uncertain.
What they tried to obfuscate by focusing on the uncertainties inherent to specific cases, was that a causal relationship of smoking and cancer \emph{had been} established.
A broad scientific consensus did not leave doubt in the fact that smoking increased the risk of developing cancer. 

Distraction can also work inversely, that is, researching beneficial effects of smoking, such as a reduced risk of preeclampsia (a condition that can occur during pregnancy \citep{lain_urinary_1999}), to then be able to claim: \textit{Smoking is not so bad it has scientifically proven benefits}.

\paragraph{3 Bad science} (Opposing single papers or methods): Again the tobacco industry: After a study by the Japanese scientist Takeshi Hirayama, finding that also non-smoking wives of smoking husbands developed lung cancer more frequently (this indicated that also second-hand smoke can cause cancer), had been published, tobacco companies funded a study that tried to discredit Hirayama's reputation \citep[p 138]{oreskes_merchants_2010}. 
They also contested an EPA report on second-hand smoke because it had relied on findings with a 90\% confidence interval, claiming that this (i.e., the reliance on relatively weak statistics) was unscientific\footnote{They also tried to seed distraction criticising that the report did not rule out other causes. (i.e., make a similar argument as the one on asbestos above.}. 
There was, however, a solid body of evidence that already showed the causal relationship of smoking and cancer. 
In such a context -- with known mechanism -- even comparatively weak scientific evidence can still make a strong case for an effect \citep[p 141-144, 156-160]{oreskes_merchants_2010}. 

\paragraph{4 Manufacturing scientific dissent} (Selective funding): \citet{abdalla_grey_2021} point out that Big Tobacco selectively funded research that was intended to shift the blame from tobacco, or research that could be used for confusion in the sense of\, \XeTeXglyph244 \,2. An example is research about whether living together with birds as pets could also be the cause of lung cancer \parencites{bero_lawyer_1995, brandt_inventing_2012}[cited in][]{abdalla_grey_2021}.

\paragraph{5 Building a cadre of experts} (Funding \textquote{promising} researchers): \citet[p 29 ff]{oreskes_merchants_2010} quote a letter on \textquote{Coorporate Support for Biomedical Research} from tobacco industry\footnote{In this case RJ Reynolds Tobacco Company.} documents: \textquote{Support [for scientific research] over the years has produced a number of authorities upon whom the industry could draw for expert testimony in courts suits and hearings by governmental bodies.} \citep{hobbs_corporate_1980} 

They also recount the case of Martin J. Cline, who was in trouble because of scientific misconduct. 
Having misrepresented a planned experiment before the authorities, he lost nearly \$200 000 in research grants.
He was subsequently funded by Big Tobacco and served later as an expert witness in many trials against tobacco companies.

\paragraph{6 Freedom \symbol{"0026} the market} (Communist over-regulation \& Costs exceed benefits): When the debate on the scientific battlefield seemed lost, some industries changed efforts and concentrated on liberal market ideologies.
They argued that institutions promoting regulation, where too strongly influenced by the political -- almost communist -- left, and their leftish desire to regulate \citep[p 166 ff]{oreskes_merchants_2010}.
They combined this claim with cost-benefit arguments about excessive costs of regulation.
For example, in the case of acid rain regulation, coal-industry lobbyists presented exaggerated estimates of the cost of regulation and exploited a weak point of cost benefit analysis, namely that often costs are comparatively easy to compute while it is virtually impossible to estimate the value of an intact environment. \citep[p 101 ff]{oreskes_merchants_2010}

\paragraph{7 \gls{pr}-campaigns} (Leveraging the Fairness Doctrine): Finally, aggressive \gls{pr} campaigns, are the strategy that combine all the previous points.
While most scientists publish their work in specialised scientific outlets, Scientists engaged by Big Tobacco and Big Oil pushed strongly into public outlets, such as \textit{The Wall Street Journal}, \textit{The New York Times} or \textit{Fortune}. 
They also published articles in outlets off affiliated institutions such as the Cato Institute or the Hoover Institution, and sent their writings directly to people, reaching a large audience. 
Some of these writings made affirmations without any reference to scientific studies.
Furthermore, they leveraged the Fairness Doctrine -- a 1949 doctrine requiring journalist to present debates of public concern in a balanced manner -- to claim equal air-time.
This created a distorted public perception (since it misrepresented the quality and quantity of scientific support) \citep[p 18 ff]{oreskes_merchants_2010}.
\vspace{\baselineskip}\\
\noindent
It is important to note that while \textit{Merchants of Doubt} lays out these techniques very clearly, they are not as easy to detect as it may seem. 
After explaining how a few scientists (among them Edward Krug) created the (false) public impression of the existence of a scientific debate around acid rain, while in essence there was none, the authors state: 
\begin{quote}
And while we are embarrassed to admit it, in the early 1990s one of us (N.O.) used Krug's arguments in an introductory earth science class at Dartmouth College to teach \textquote{both sides} of the acid rain \textquote{debate}.
\end{quote} 
\citep[p 103]{oreskes_merchants_2010} 