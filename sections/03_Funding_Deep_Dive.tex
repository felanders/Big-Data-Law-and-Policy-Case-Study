\section{Facebook Funding Deep Dive}
\label{sec:fac_fund}

As pointed out in the previous section, understanding Facebook's funding policy is difficult. As argued in \citep{abdalla_grey_2021} funding plays an important role, thus I want to do a deep dive on FB's funding policy on a prominent example:

After increasing public concern about polarization, FB\marginnote{This is the related \href{https://research.fb.com/programs/research-awards/proposals/foundational-integrity-research-misinformation-and-polarization-request-for-proposals/}{FB site}} \textquote{announced USD 2 million in funding for independent research proposals on polarization} \citep{seetharaman_facebook_2020}. 

In this section I want to scrutinize this specific funding program in the following dimensions:
\begin{itemize}
    \item How congruent are the topics set by FB for the funding with the problems of polarization on FB claimed by scientists and the public?
    \item How transparent are funded researchers about the funding made: Is it visible on their academic CV's and is it mentioned in related papers?
    \item How congruent are the projects actually funded with the underlying problem?
\end{itemize}

For this I plan to create a list of the topics set by Facebook  and analyze them in terms of their congruence.
Furthermore, I want to create a table with the researchers listed on \href{https://research.fb.com/programs/research-awards/proposals/foundational-integrity-research-misinformation-and-polarization-request-for-proposals/}{the related FB site} and rate them on transparency, assign their project to a topic if possible and asses the congruence with the overall goal.

