\begin{center}
    \Large\sffamily{\textbf{Abstract}}
\end{center}
%\addcontentsline{toc}{section}{Abstract}
\begin{abstract}
\noindent
We know from the historical record that, powerful industry players (among others from the tobacco, oil, and agrochemical industries) were actively sowing doubt about inconvenient scientific findings.
Only recently, big (digital) technology firms (Big Tech) have grown to become the world's most valuable companies, and been confronted with the problems created by their technologies.
The claim that many digital services are built to be addictive, is one of the most prominent points of critique of Big Tech, and Facebook more specifically. 

In this light, the present case study maps the debate on social network service addiction and finds a complicated field of study. While there is debate about how to call the issue and which methodologies to use, there seems to be a broad body of evidence showing connections between excessive social media use and negative effects on people. 
The paper also investigates Facebook's position on the issue and identifies some techniques used by Big Tobacco and others.
While there is not enough evidence to accuse Facebook of sowing doubt, the company is certainly \emph{leveraging uncertainties} in the science on social network addiction.

A deep dive on one Facebook Research award on misinformation and polarization, examines how transparently researchers disclose the funding, and how congruent the corporate communication is with the actual projects.
For more than 50\% of the researchers, it was not possible to find a disclosure of the grant on their personal web presence or CV, almost one year after the award was granted.
Moreover, some projects can be suspected to not addressing the topic announced publicly, but target issues that could create talking points in favour of Facebook.

Concluding this paper presents some implications for law and policy, highlights limitations of the study, and presents some open questions.
\end{abstract}