\section{Problematic Use of Social Network Services}
\label{sec:prob_sns}
%\addcontentsline{toc}{section}{Problematic Use of Social Network Services}

\subsection{In the Scientific Debate}
\begin{enumerate}
    \item Wording: How is the issue talked about in science?\begin{itemize}
        \item Addiction vs. Problematic use
        \item What are the reasons for using both terms?
    \end{itemize} 
    \item Scientific findings on (problematic) SNS use (focusing on meta studies): \begin{itemize}
        \item Correlations with different variables
        \item Relation to negative effects on well-being
    \end{itemize} 
    \item The open secret (FB would not obṕenly state this) that Facebook is maximizing user-engagement \begin{itemize}
        \item User engagement is code for time spent on Facebook
        \item There is many reports that Facebook prioritizes this above almost anything else (also polarization, depression...) e.g. interviews, and internal documents.
        \item Scientificly studied mechanisms keeping users \textquote{engaged} are two patterns mentioned in \citep{montag_addictive_2019} namely endless scrolling and showing people what they like\footnote{I know that some people claim that this in combination with random gratifications is especially powerful, however, I still have to find research on this. Also notifications (esp. frequency) seem to play an important role for engagement.}.
    \end{itemize}
    \item Limitations: \begin{itemize}
        \item Correlation vs. Causation (lots of FB => bad mood vs. Bad mood => lots of FB.)
        \item Funding of researchers is often not easily found, this makes it difficult to understand which research was under influence of FB => Section \ref{sec:fac_fund}
        \item It is very time consuming / beyond the scope of this case-study to map out the whole scholarly debate and understand how the field evolved over time i.e. understand which papers/authors where influential and how they shaped the following research ... This could be an interesting analysis for a future case study e.g. building a citation graph.
    \end{itemize} 
\end{enumerate}


\subsection{Facebook's Part in the Debate}
\begin{enumerate}
    \item Facebook avoids talking explicitly about the fact that its aim is to maximize users engagement (i.e. time) on their service.
    \item One important argument is that FB brings many advantages you just have to use it right:\begin{itemize}
        \item They present research on this on their blog: active FB usage i.e. messaging makes people feel better; emotional contagion can also happen via FB \citep{kramer_experimental_2014}\footnote{The paper Facebook got a huge backlash for, because they did not asked for consent and also made people feel worse.} (one interpretation of the research could be that they where trying to indicate that FB \textit{can} be a substitute for real world interaction)
        \item They held a panel during their 2019 global safety and well-being summit \citep{groman_so_2019} where this argumentation was predominant.
        \item Moreover on their blog they argue that Facebook can help you in situations of crisis: If you feel depressed just write a message to a friend \citep{facebook_connecting_2020, davis_connecting_2017, facebook_making_2021} the message goes.
    \end{itemize}
    \item Another argument is that there is not enough evidence or evidence is not good enough: e.g. in \citep{groman_so_2019}. 
    \item A last argument that is made more subtle is that everyone is responsible themselves and FB should not restrict its users. Again brought forward in \citep{groman_so_2019} but also the underlying idea of \citep{facebook_connecting_2020, davis_connecting_2017, facebook_making_2021}.
    \item Limitations: \begin{itemize}
        \item It is very difficult to disentangle the effects of smartphones and SNS, and understand where exactly problematic use originates/how it can be tackled. That is, it is difficult to argue against point 2 in this list.
        \item It is rather difficult to assess Facebook's position over time: their newsroom page is not too helpful: e.g. searching for "problematic use" results only in 14 entries as of today. Of these 14 posts none is addressing problematic use of Facebook or one of its other services. Moreover, the oldest entry is from Sept. 21st 2017.
        \item Contrary to the case of Big Tobacco there is no extensive insight into the exact information FB internally has (in terms of scientific findings). There are only few leaks and reports of former employees.
        \item Finally, finding out about Facebook funding is difficult on a larger scale (Finding out whether researchers where funded by Facebook is difficult, also finding out what projects FB funded is not straightforward).
    \end{itemize}
\end{enumerate}