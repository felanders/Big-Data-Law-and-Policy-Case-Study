\section[Social Network Addiction]{Social Network Addiction\footnote{As outlined below, there exists a debate about whether to call compulsive SNS use an addiction or problematic use. I am aware that using one term instead of the other implies taking a stance on the debate. My reasons for using SNS addiction here rather than problematic SNS use is that the debate revolves around the question whether SNS are addictive or not, thus the literature is \emph{about} SNS addiction, while it still might find that there is no such thing as SNS addiction. In the following, when reporting on research articles, I will use the term that the authors used.}}
\label{sec:sns_add}

The term \textit{\gls{sns} addiction} is frequently found in popular media (sometimes with Twitter, Instagram, Facebook, or the like instead of \gls{sns}). 
Often these reports relate to, or present parts of, the scientific body of evidence on \gls{sns} or internet addiction. A few examples are: \citep{rosen_smart_2019, flores_ex-facebook_2020, flanagan_you_2021, john_too_2021, duff_im_2021,  xue_science_2021}.
Note that some of these articles reject the classification of the problem as an addiction, \citet{john_too_2021} for example claims that \gls{sns} are not addictive like drugs, or \citet{rosen_smart_2019} argues that often not all symptoms required to classify an addiction are met and that the problem can be better understood in through the concept of anxiety.
This raises the question: 
\begin{quote}
    \textit{Does social network service addiction exist? And if yes, can we measure it?}
\end{quote}
Comprehensively understanding and describing the research and debate about \gls{sns} addiction is a difficult task, since it encompasses discussions about the use of specific terminologies and has overlaps with research into addiction to smartphones, video-games, and the internet in general.
Furthermore, doing a systematic literature review of the field is beyond the scope of this paper.

In this light, I will focus on literature reviews about \gls{sns} addiction and present dissenting views expressed in academic journals or in popular media outlets to better understand the points of discussion.

The basic approach to finding relevant literature was of iterative nature.
That is the starting point was an unstructured search on \href{https://pubmed.ncbi.nlm.nih.gov/about/}{PubMed}, \href{https://www.webofscience.com/}{Web of Science}, and \href{https://www.semanticscholar.org/}{Semantic Scholar} to identify relevant literature. 
This approach was complemented with Google searches for media referring to the science of social media or social network addiction, and a search for papers connected to the ones identified previously through \href{https://www.connectedpapers.com}{Connected Papers}. 
This process was iterated several times. 
The main criteria for inclusion of papers were: impact (among others citations, but also relevance for the topic), and novelty of arguments (in order to map the important points of the debate).

\subsection{Mapping the Scientific Debate}
\label{sec:mapping}
\subsubsection{The predominant stream of research}
Most of the early work on social media addiction builds upon work on internet addiction. 
\citet{young_internet_1998} wrote an influential early paper on internet addiction, where the author draws a comparison between internet addiction with gambling addiction.
She found (using an adapted gambling addiction questionnaire) that internet dependents spend almost 8 times as much time on the internet (i.e., 38.5 $\pm$ 8.04 self reported hours per week), than non dependents. 
She also reports about anecdotal consequences of problematic internet use such as: negative impacts on social relationships, financial problems, work related problems, and mild physical complaints.
Furthermore, the study reported subjects having problems to control the internet use, and showing withdrawal symptoms (cravings) when internet use was stopped. 
In her conclusion, the author highlights not only the limitations of her study -- which suffers multiple biases -- but also the fact that the internet itself is not addictive, however, specific applications could be.

The article by \citet{kuss_online_2011} in 2011 was the first systematic literature review on \gls{sns} and addiction. 
The authors motivate their work by qualitative studies of anecdotal evidence that indicate potentially addictive features of SNS.
In describing the field, they also refer to scholarly work on internet addiction (among others also the work by Young).

\citet{kuss_online_2011} report among other aspects also studies finding relations of excessive SNS use and: decreased self-esteem, or reduced academic achievement (in terms of \gls{gpa}).
According to the authors, however, the evidence did not permit any conclusive statements about negative impacts of SNS, by that time.

Exploring the potential of SNS addiction \citeauthor{kuss_online_2011} define symptoms of \gls{sns} addiction according to \citep{griffiths_components_2005} as:
%\begin{addmargin}[2em]{4em}
\begin{quote}
\begin{itemize}
    \item [-] [...] \textit{mood modification} (i.e., engagement in SNSs leads to a favourable change in emotional states),
    \item [-] \textit{salience} (i.e., behavioral, cognitive, and emotional preoccupation with the SNS usage),
    \item [-] \textit{tolerance} (i.e., ever increasing use of SNSs over time),
    \item [-] \textit{withdrawal symptoms} (i.e., experiencing unpleasant physical and emotional symptoms when SNS use is restricted or stopped),
    \item [-] \textit{conflict} (i.e., interpersonal and intrapsychic problems ensue because of SNS usage),
    \item [-] [...] \textit{relapse} (i.e., addicts quickly revert back in their excessive SNS usage after an abstinence period).
\end{itemize}
\end{quote}
%\end{addmargin}
\citep[p 3530][itemisation and emphasis added]{kuss_online_2011} 

These definitions are derived from classical definitions of addictions and widely used in the field.

The work by \citet{andreassen_development_2012} builds upon \citep{young_internet_1998, griffiths_components_2005, kuss_online_2011} and presents -- based on the definitions presented above -- the \gls{bfsa} which was the first attempts to measure Facebook addiction.
This article sparked a discussion about whether a Facebook addiction scale or a Social Media addiction scale is more needed for the advancement of science\footnote{See: The comment of \citet{griffiths_facebook_2012} and the reply of \citet{andreassen_facebook_2013}.}. 

In 2012, \citet{kuss_internet_2012} conducted a literature review of studies using  neuroimaging\footnote{That is, techniques to make pictures of the brain in action. 
Specifically, they reviewed studies using \gls{mri}, \gls{pet}, \gls{spect} for brain imaging. 
They also reviewed studies using \gls{eeg} which records the electrical activity on the scalp.} to study internet- and gaming-addiction.
The findings they summarise, link excessive internet use and gaming addiction to changes in the dopamine activity (which is decreased in \textquote{addicts}) and report neuroadaptation as result of increased activity in certain brain regions, that is structural and functional changes in the brain.

Interestingly, the authors use the term \textit{internet addiction}, as if it was an established disorder. 
They also mention that the American Psychiatric Association had included \textquote{\emph{Internet use Disorder} as mental health problem worthy of further scientific investigation} \citep{kuss_internet_2012} to the fifth version of its \gls{dsm} \citep{american_psychiatric_association_diagnostic_2013}.
This inclusion into topics of further scientific investigation was the result of a debate around the inclusion of internet use disorder into the \gls{dsm}-V.
\citet{block_issues_2008} for example, proposed the inclusion of internet addiction into the \gls{dsm}-V, while \citet{weinstein_internet_2010} rejected this idea. 

After internet addiction was not included in the -- by then -- most recent version of \gls{dsm} (i.e., \gls{dsm}-V), \citet{ryan_uses_2014} focused their research on the newly emerging field of \gls{sns} addiction\footnote{The reports in figures \ref{fig:internet_addiction} and \ref{fig:sns_addiction} in appendix \ref{app:figures} show the number of publications related to internet (\ref{fig:internet_addiction}) and SNS (\ref{fig:sns_addiction}) addiction. It can be observed that in the year 2013 -- after internet addiction was not included in \gls{dsm}-V -- the number of publications on SNS addiction increases more rapidly than the number of publications on internet addiction. While speculative, this could indicate a shift in research priorities from internet to SNS addiction. Since after the  \textquote{death} of internet use disorder as \textquote{official} disorder, this field might have become more promising for researchers. At the same time, the data could simply show a correlation with the increasing spread of SNS.}.
Building among others on \citep{kuss_online_2011, kuss_internet_2014}, they conduct a literature review of 24 studies focused on Facebook, and report that this focus was novel in the field.

In their introduction, the authors appeal to increasingly focusing research on specific gratification mechanisms.
They argue that only this way, addictions to different services could be understood. 
For example, they propose separate analysis of Twitter and Facebook, and even a differentiation between services on Facebook such as: Games like the by then popular \emph{Farmville} (falling under \textquote{gaming addiction}), or Facebook's feature to post content (falling under \textquote{cyber-relationship addiction}).

Moreover, the authors differentiate between two important, and often misconceived, concepts related to addiction:
\begin{itemize}
    \item \emph{Habitual use:} relates to using Facebook without concrete reason or unconsciously accessing the service. Habitual use is described as a potential risk factor for developing a Facebook addiction.
    \item \emph{Excessive use:} relates to spending large amounts of time on Facebook or accessing the service many times a day. The authors emphasise that spending much time on Facebook or -- in reference to internet addiction research -- the internet, does not necessarily indicate an addiction.
\end{itemize}
Additionally, they point out a measurement problem. That is, measurements of Facebook addiction are not based on a common definition, which impairs comparison of studies and meta-studies.

In their conclusion, the authors again emphasise the importance of studying gratifications for understanding Facebook addiction.
Moreover, they indicate the tension and complexity emerging from the interconnection of beneficial uses with gratifications.
Since the gratifications are inherent to some services, and they might induce an addiction, differentiating causes and effects can become difficult.
Furthermore, they describe a mechanism of low psycho-social well-being (\textquote{loneliness, anxiety or depression} \citep{ryan_uses_2014}) leading to increased Facebook use, which lifts the mood in the short term, however, can in the long term amplify the initial condition.

\citet{andreassen_online_2015} presented a concise overview of the field in 2015, basing her review especially on \citep{andreassen_development_2012, kuss_internet_2014, andreassen_social_2014, ryan_uses_2014, kuss_internet_2014} and trying to synthesise the findings in these studies.
She mentions epidemiological studies on \gls{sns} addiction (That is, studies about how widespread \gls{sns} addiction is) and concludes that given the lack of cross-cultural and longitudinal\footnote{i.e. studies over an extended time-period.} studies, together with methodologies varying across studies, and non-representative samples (with a strong focus on students): \textquote{[...] it is premature to draw conclusions about prevalence and relevant risk factor of SNS addiction.} \citep{andreassen_social_2014}
The author reiterates the problem of inconsistent measurements and goes on to outline three possible explanations for the development of \gls{sns} addiction:
\begin{itemize}
    \item \textit{Dispositional factors}: \begin{itemize}
        \item \textit{Neurobiology}, seems to play an important role in the development of (\gls{sns}) addictions. Thus, neurobiological dispositions could be one explanation for the development of a \gls{sns} addiction.
        \item \textit{Personality factors}, (especially within the framework of the Big Five personality traits\footnote{The five categories used to quantify personality in this model are: \textquote{neuroticism, extroversion, openness to experience, agreeableness, and conscientiousness} \citep{andreassen_online_2015}.}) have been shown to correlate with \gls{sns} addiction. Especially neuroticism and extroversion predict \gls{sns} addiction, while conscientiousness negatively correlates with \gls{sns} addiction.
        \item \textit{Innate basic psychological needs}, such as needs for: \emph{competence} (i.e., being able to use \gls{sns}), \emph{autonomy} (in terms of expression) and \emph{relatedness} (i.e., being connected with others) or a distortion of them, could explain the formation of compulsive \gls{sns} use and this way lead to addiction.
        \item \textit{Basic cognitions}, such as a core belief of not being good enough, could for example lead to an obsession with likes on \gls{sns}.
    \end{itemize}
    \item \textit{Sociocultural factors}: such as the context we live in and role models we observe could foster problematic \gls{sns} use.
    \item \textit{Behavioural reinforcement factors}: Mechanisms described in the theories of operant conditioning (e.g., through nearly instant feedback) and social learning could also contribute to explaining the development of \gls{sns} addiction.
\end{itemize}

In the following section the author elaborates on research about negative the consequences of problematic \gls{sns} use, which she categorises into the following effects:
\begin{itemize}
    \item \emph{Emotional problems}, such as negative feelings, can lead to a vicious cycle as explained above. Furthermore, the author reports on research finding correlations between \gls{sns} addiction and depression, anxiety, low self-esteem and low well-being.
    \item \emph{Relational problems}, have also been found to correlate with \gls{sns} addiction. Here a similar mechanism as with emotional problems could be at work. That is, excessive \gls{sns} use leads to relationship problems, which in turn lead to compensation by even more excessive \gls{sns} use.
    \item \emph{Health related problems}, like sleep difficulties and a lack of exercise, have been found to correlated with \gls{sns} addiction in some studies. 
    \item \emph{Performance problems}, both work related and academic, have been found to correlated with \gls{sns} addiction. The author adds to this point that given the ubiquitous presence of the internet in professional and personal life, the aim must be to control problematic \gls{sns} behaviour rather than trying to completely abstain from \gls{sns} or the internet.
\end{itemize}

Finally, \citet{andreassen_online_2015} explores research on approaches for \gls{sns} addiction therapy and finds relatively little reliable research.
In the conclusion, she highlights the complexity of the topic and the debate about whether to frame research in terms of general \gls{sns} or in terms of specific services.
In a closing appeal, the author calls for more reliable research, especially cross-cultural, and longitudinal studies with representative samples. 

\citet{baker_relationship_2016} presented a systematic review of 30 quantitative studies on depression and \gls{sns} in 2016.
The bottom line of the article is -- like the infamous relationship status on Facebook -- \emph{It is complicated!} There are many different ways to use social networks, and these differences in time, quality and type of use can influence how the \gls{sns} impacts people. 
That is, specific ways of using \gls{sns} can have a positive impact on people, while others might affect subjects negatively. 
Furthermore, so the authors report, many social, psychological, and personal factors in this rapidly seem to play into the outcomes, this together with the rapidly changing environment, makes it difficult to single out and reliably measure specific effects.
They also mention weaknesses of current research, such as a strong bias towards Facebook and cross-sectional\footnote{That is, studies focusing on a sample of participants at one point in time.} studies.

The authors conclude, that while claiming that Facebook causes depression is overly simplistic, and there is research measuring benefits of \gls{sns}, \textquote{the findings suggest that for some people online social networking may be associated with increased symptoms of depression} \citep{baker_relationship_2016}.
This is why the authors are convinced that more, better (e.g., longitudinal studies) and more specific research (e.g., studying specific gratification mechanisms) is needed. 

Finally, in 2017 \citet{kuss_social_2017} summarised the state of \gls{sns} addiction research in the following ten lessons learned:
%\begin{addmargin}[2em]{4em}
\begin{quote}
\begin{itemize}
    \item[(i)] social networking and social media use are not the same;
    \item[(ii)] social networking is eclectic; [i.e., a SNS can be many things]
    \item[(iii)] social networking is a way of being; 
    \item[(iv)] individuals can become addicted to using social networking sites;
    \item[(v)] Facebook addiction is only one example of SNS addiction;
    \item[(vi)] \gls{fomo} may be part of SNS addiction; 
    \item[(vii)] smartphone addiction may be part of SNS addiction; 
    \item[(viii)] nomophobia  [(\acrlong{nomo}) i.e., the fear of not having access to one's smartphone] may be part of SNS addiction;
    \item[(ix)] there are sociodemographic differences in SNS addiction; [...]
    \item[(x)] there are methodological problems with research to date.
\end{itemize}
%\end{addmargin}
\end{quote}
\citep[formatting added]{kuss_social_2017}

After this paper, researchers seemed to follow the calls of previous papers, and study the specifics of problematic \gls{sns} use. 
Some examples of research papers influenced by \citep{kuss_social_2017} are:
\citep{balakrishnan_social_2017} researches the influence of YouTube content on \gls{sns} addiction;
\citep{monacis_exploring_2017} studies individual differences (the role of identity and attachment on \gls{sns} addiction;
\citep{kircaburun_instagram_2018} studies Instagram addiction and personality traits; 
\citep{kircaburun_uses_2020} focuses on gratifications, uses of \gls{sns} and their relation to personality traits; 
\citep{balcerowska_is_2020} asks: \textquote{Is it meaningful to distinguish between Facebook addiction and social networking sites addiction?};
\citep{fabris_investigating_2020} looks at the connection between FOMO and \gls{sns} addiction.

After \citeauthor{kuss_social_2017}' \textquote{ten lessons learned}, \citet{darienzo_addiction_2019} presented the only systematic literature review related to the field that I could find.
In their article, the authors review 32 studies on attachment styles, and \gls{sns} or Internet addiction.
Attachment here refers to children's relation with their attachment person, that is, the most important figure in their lives. 
Attachment theory states that there are four different types/styles of attachment and that these stay fairly stable over time, because the ways to interact in relationships are to some extent internalised and reproduced.

The main finding of the report is that Facebook and other \gls{sns}s can serve as a compensation or replacement for a lack of care or strong social relationships.
The authors also state that in such cases, research indicates that \textquote{having a high engagement online constitutes a vicious cycle for those individuals, because they will be at high risk of social withdrawal.} \citep{darienzo_addiction_2019}
Finally, they conclude by acknowledging that self reporting and missing longitudinal studies are limiting factors.

\subsubsection{Points of controversy}
\label{sub:contr}
As it is natural in science, there has also been criticism on some of the previously mentioned works. 
Since disagreement often helps to understand an issue better, I will outline some major points of controversy.
Some of these arguments have been made in outlets aimed at a more general public, that is, they did not directly address the scientific community.

\paragraph{Addiction vs. problematic use?} As pointed out in \citep{ryding_internet_2018}, there is a vivid discussion around the naming of the problem at hand, which is almost as old as the research on the problem itself. Within this debate, various arguments have been put forward:

\subparagraph{DSM-V inclusion} The \acrlong{dsm} can be described as a collection of what the psychiatric community agreed to call a disorder. 
In 2008 \citet{block_issues_2008} proposed the inclusion of internet addiction into the DSM-V while \citet{weinstein_internet_2010} rejected the idea. 
Finally, internet addiction was not included in the DSM-V in 2013 \citep{ryan_uses_2014}.
Also, \gls{sns} addiction is neither mentioned in the DSM-V nor in the \gls{icd} the manual on all (not just mental) disorders published by the \gls{who} \citep{who_icd-11_2019}.
This indicates, that there is no consensus or clear-cut definition of \gls{sns} addiction.
For some authors, this is an argument to talk about problematic use instead of \gls{sns} addiction \citep{rosen_obsessiveaddictive_2018, john_too_2021}. 

Some arguments also referred to the fact that internet gambling was not included in the \gls{dsm}-V, and should thus also not be considered an addiction \citep{rosen_smart_2019}. 
However, this is subject to change, since the eleventh revision of the \gls{icd} will contain online and offline gaming as a \textquote{Disorders due to substance use or addictive behaviours} \citep{who_icd-11_2019} when coming into effect on January 1st, 2022 (it was endorsed on May 25th, 2019).
By analogy of mechanisms, \citep{kuss_social_2017, ryding_internet_2018} this might lend \gls{sns} addiction research some legitimacy.

\subparagraph{Overpathologisation} Some authors argue that introducing \gls{sns} addiction as a disorder could lead to overpathologisation (i.e., to declare more than necessary people as abnormal). 
To make their point, they argue that research is inconclusive -- or not specific enough -- \citep{howard_analysis_2015, vuorre_there_2021,robertson_why_2021} 
and that declaring problematic \gls{sns} use as an addiction would put it into the same bucket as substance abuse (of e.g., \textquote{heroin, cocaine or alcohol} \citep{carbonell_critical_2017}) \citep{john_too_2021, rosen_smart_2019}.
\citet{carbonell_critical_2017} advocate for not blurring the lines between what they call \textquote{true addictive disorders, [i.e. ...] substance addiction} and \textquote{negative side-effects of engaging with certain appealing activities like SNSs}, in order not to undermine the severity of psychiatric disorders. 
In a later article, the same authors make similar arguments about smartphone addiction \citep{panova_is_2018}.

\subparagraph{Are there better explanations?}
There are some researchers promoting other concepts such as anxiety or obsession to explain \gls{sns}s negative effects on well-being \citep{bragazzi_proposal_2014, buglass_motivators_2017, oberst_negative_2017, rosen_obsessiveaddictive_2018, panova_specific_2020}. 
Especially, the concepts of anxiety and obsession might help explain some of the observed effects.
Anxiety in the \gls{sns} context often refers to FOMO and nomophobia, which have also been identified by \citep{kuss_social_2017} as important factors which play into their concept of \gls{sns} addiction.

\subparagraph{General unreasonable fear of new technology} 
A more general argument in the debate is that we tend to be overly fearful when encountering new technologies, and expect dangers where there are none \citep{ryding_internet_2018, orben_sisyphean_2020}. 
This argument seems appealing, however, it is difficult to prove or disprove.

\paragraph{Methodological flaws} As mentioned throughout the discussion above, there are methodological concerns regarding many studies in the field. 
While researchers were often aware of these downsides, they still limit the reliability of the results.
\subparagraph{Representativeness} of the samples is one such concern, since many studies relied on university students or young adults as participants and comparatively small sample sizes. 
To defend the research, one could argue that while generalising such findings is not possible, the results are still relevant since young adults and adolescents are the heaviest \gls{sns} users. 

\subparagraph{Correlation vs. causation} Most of the studies reported, where cross-sectional and did not follow individuals over time. 
Thus, research faces the problem of not knowing for example whether depression caused individuals to become addicted to \gls{sns} or whether \gls{sns} addiction caused depression \citep{carbonell_critical_2017}.

\subparagraph{Self reporting} Much criticism was also put forward to the reliance on self reporting to measure addiction \citep{geyer_absence_2021}. 
A recent systematic review in \emph{Nature Human Behaviour} \citep{parry_systematic_2021} found  little to no correlation between reported and actual digital media use. 
That is to say that self reported social media use is not useful to predict actual digital media use. 
Similarly, \citet{orben_screens_2019} found that there is little to no correlation between screen time and well-being.

\paragraph{Policy Change} There, are also a number of studies addressing the conclusiveness of evidence and potential benefits of \gls{sns}, which both play into policy decisions.

\subparagraph{Evidence}
For example, \citet{jelenchick_facebook_2013} \textquote{[...] did not find evidence supporting a relationship between SNS use and clinical depression} and think counselling on \textquote{[...] the risk of “Facebook Depression” may be premature.}
Or a study by \citet{orben_association_2019} on well-being and digital technology use stated: \textquote{Taking the broader context of the data into account suggests that these effects are too small to warrant policy change.}
Finally, a 2020 longitudinal study over 8 years with participants filling in a yearly survey found no association between mental health issues and \textquote{time spent on social media} \citep{coyne_does_2020}.

Conversely, however, a 2016 cross-sectional survey on 1,787 US adults ages 19 to 32 found: \textquote{SM use was significantly associated with increased depression. Given the proliferation of SM, identifying the mechanisms and direction of this association is critical for informing interventions that address SM use and depression.}\citep{lin_association_2016}

\subparagraph{Benefits} 
A study on well-being of new mothers found positive effects of blogging \citep{mcdaniel_new_2012}, while a study on Facebook use and clinical symptoms of psychiatric disorders found \textquote{positive and negative aspects of technology [... and ...] apparently detrimental effects of a preference for multitasking.} \citep{rosen_is_2013}.
More recent studies also tried to measure the value of social networks through the \gls{wta} stopping to use Facebook. 
\citet{corrigan_how_2018} estimated the value of one year of Facebook (i.e., the deactivation of it) to be more than \$1,000, whereas \citet{allcott_welfare_2020} reported a median and mean \gls{wta} deactivating Facebook for 4 weeks of \$100 and \$180 respectively. 
However, the \gls{wta} declined after participants had not used Facebook for four weeks, which suggests \textquote{[...] that traditional metrics may overstate consumer surplus} \citep{allcott_welfare_2020}.

Finally, \citet{kross_facebook_2013} state: \textquote{On the surface, Facebook provides an invaluable resource for fulfilling the basic human need for social connection. Rather than enhancing well-being, however, these findings suggest that Facebook may undermine it.}

\subsection{Summary} 
In summary, we have seen how the research on \gls{sns} addiction developed, building on research about \textit{internet addiction}.
While during early internet addiction research, 38 hours of internet use per week were considered highly problematic, today many of us exceed this number. 
This shows that the \textit{social context}, and the \textit{type of use,} both influence the perception of certain behaviours as an addiction.

Very early on, internet addiction scholars were already aware, that the internet per se is not addictive, but that \textit{specific services} on the internet are -- an insight that was reiterated several times.
Furthermore, \textit{negative effects} on relationships, finance, and work related problems were reported to correlate with excessive internet use.
In the 2010s, researchers started to investigate \gls{sns} and their potential addictiveness.
This shift coincided with the American Psychiatric Association's decision not to include internet addiction in the \gls{dsm}-V.

Going further, researchers explored neurobiological mechanisms, as well as \textit{gratification mechanisms}, personality types, sociocultural factors, and negative impacts on peoples lives. 
Throughout this effort, studies suffered from three common \textit{weaknesses}.
That is: they predominantly relied on \textit{self-reporting}; they used various \textit{different measurement scales}; and they were predominantly \textit{correlational}.

These limitations were aggravated by difficulties to characterise social networks. 
Since these platforms consist of a vast number of services and features, coming to conclusive statements was difficult. 
Even more so, because of the many ways to use all these different services and the consequence that 
\textit{simple measures} like or time spent or the number of interactions are \textit{not able to capture this complexity}. 
That is, excessive or compulsive use is difficult to define -- and thus measure -- on the basis of such numbers.

However, some findings seemed to be fairly consistent throughout the development of the field.
Especially, the fact that \textit{individuals can become addicted} to \gls{sns} and that excessive \gls{sns} or internet use can have \textit{detrimental effects} on peoples lives (e.g., on social relationships, anxiety, or depression). 
Furthermore, researchers consistently argued that \textit{gratification mechanisms} are important for understanding addiction.

\subsection{Facebook's Position}
Since Facebook is a private company, it will most probably try to avoid topics that cast shadow on the company. 
However, there are some occasions where the company or its representatives did. 
Moreover, analysing the picture that the company paints of itself, set in relation to reports about internal practices, will help to address the question of:

\textit{How does Facebook relate to this issue, in the light of the research presented above?}

There is generally few official statements of Facebook's position on the issue of addiction or problematic use: 
Their \href{https://about.fb.com/news/}{newsroom page} (Facebook's own \gls{pr} outlet) features specific sections (topics) on: Data and Privacy, Technology and Innovation, Safety and Expression, Election Integrity, Combating Misinformation, Economic Opportunity, and Strengthening Communities, however none of these topics covers issues related to problematic \gls{sns} use.

Searching their newsroom page for \textquote{problematic use} results in only 14 entries, dating back to September 21st, 2017 the latest (searching for \textquote{addiction} returns 7 unrelated posts).
Of these 14 posts, none is actually addressing problematic Facebook use or problematic use of other Facebook services (Instagram et al.).

The only post that is somehow related to problematic use introduces the expansion of feedback surveys, trying to improve user's news-feed by learning what they want to see and prioritising this in the algorithm.
Peculiarly, this post -- found by searching for problematic use -- ends on: \textquote{While engagement will continue to be one of many types of signals we use to rank posts in News Feed, we believe these additional insights can provide a more complete picture of the content people find valuable, and we’ll share more as we learn from these tests.} \citep{gupta_incorporating_2021}

An important strategy of Facebook with regard to problematic use and addiction is to point out Facebook's good sides.
For example, the -- by now infamous\footnote{The researchers did not ask users for consent, and exposed some of them to more negative posts, which lead to broad criticism.} -- study by \citet{kramer_experimental_2014} showed that emotional contagion (i.e., sharing of emotions) can also happen trough Facebook. 
This could be interpreted such that Facebook \emph{can}, to a certain extent, replace real interaction.

Research presented on Facebook's news page perpetuates the argument. 
For example, \citet{ginsberg_hard_2017} present studies finding that passive interaction on Facebook can lead to negative impacts on users' mental health, while selected other studies (among others from Facebook\footnote{One of these researcher is Robert E. Kraut, Professor at Carnegie Mellon University and Academic Researcher at Facebook. Kraut who frequently collaborated with Moira Burke (Research Scientist at Facebook) published also work on Facebook and did at least in two occasions not make his affiliation with Facebook transparent. See: \citep{burke_using_2013}, \citep{burke_relationship_2016} and his vita \citep{kraut_vita_2020} }) shows that meaningful and active engagement on Facebook can increase peoples' well-being.
This is the basis for a central argument of Facebook: If Facebook is only used the \textit{right} way, then it is beneficial to you. 
Following, the authors report about Facebook's efforts to improve the news-feed, give people the option to take a break from Facebook or specific users, and to provide information for suicide prevention, in order to make Facebook a place of more meaningful interaction.

Also, statements about mental health follow similar lines of argumentation.
In the blog posts related to these issues, Facebook reports on measures to make mental health resources more easily available, as well as measures like special stickers that supposedly help users reaching out to a friend in case they experience emotional or mental trouble \citep{davis_connecting_2017, facebook_connecting_2020, facebook_making_2021}.

Following up on \citep{ginsberg_hard_2017} and well in line with Facebook's official arguments, Mark Zuckerberg (foun\-der and CEO of Facebook) announced in 2018 that given the research findings \textquote{I'm changing the goal I give our product teams from focusing on helping you find relevant content to helping you have more meaningful social interactions.} \citep{zuckerberg_one_2018}

Another event that shed light on Facebook's positions to problematic use/addiction was the hearing of Mark Zuckerberg before the \textit{US Senate Committee on the Judiciary} and the \emph{US Senate Committee on Commerce, Science, and Transportation} under the title \emph{Facebook, Social Media Privacy, and the Use and Abuse of Data}.
The hearing was part of the investigations around the potential influence of Russia in the 2016 US presidential election.

During the hearing, there were only two questions by Senator Ben Sasse (NE-R) addressing Facebook addiction. 
First, Sen. Sasse asked: \textquote{As a dad, do you worry about social media addiction as a problem for America's teens?} \citep[at 3:15:40]{noauthor_facebook_2018}.
The response of Mr. Zuckerberg was in line with previous statements and in essence meant:
Facebook is doing its best to avoid negative side effects; Research has found that if Facebook is used to strengthen relationships, users benefit, and Facebook aims to engage users to do so.
Second, Sen. Sasse asked whether Facebook engaged consultants to learn how to create dopamine feedback loops to get users hooked, which Mr. Zuckerberg denied.

After the hearing, a picture of Mr. Zuckerberg's notes was published \citep{stefan_becket_photo_2018}.
These notes included important talking points and augmentations. 
The section on well-being is unfortunately partially covered in the picture. 
While being speculative, an argument that would fit the visible word pieces and Facebook's \gls{pr} would be that Facebook promotes time well spend, not simply time spent on Facebook. 

Furthermore, questions that could not be answered for time reasons during the hearing were posed to Facebook and answered by the firm in written form. 
Facebook's response to a question by Senator Dan Sullivan (AK-R) asking among others about concerns of \textquote{[...] children's increased use, [...] that rises to level of addiction [...]} did not substantially address addiction or problematic use. 
The response focused on security, safety, and education  and mentioned tools to increase parents control over the content their children see and the time they spend online \citep[p 27]{facebook_inc_post_2018}.

To the question if Facebook funds \textquote{[...] any research on the issue of potential social media addiction, [...]}, Facebook responded by highlighting that the company employs and cooperates with \textquote{social psychologists, social scientists, and sociologists [...] to better understand well-being}. Furthermore, they highlighted that Facebook recently pledged a research fund of \$1 \gls{mn} on this topic \citep[p 210]{facebook_inc_post_2018} \citep{davis_hard_2017}.
Unfortunately, no information about this award is available on the website of Facebook Research to date \citep{facebook_research_research_nodate}.

Finally, during a panel on Facebook's 2019 global safety and well-being summit, invited researchers promoted similar arguments \citep{groman_so_2019}.
The main lines of reasoning where: Facebook brings large benefits to its users if it is used the right way; Users do and should have the freedom of choice about how to use the internet or Facebook, and there should be no unnecessary restrictions to this. Regulation should also be avoided because there is not enough scientific evidence to warrant state governed alterations of the service. 

\subsection{Contextualising the Debate}
\paragraph{Unpacking Engagement}
As mentioned in the introduction, there have been multiple former Facebook employees reporting that the company is optimising its service to maximise one target: \textit{user engagement}. 
There are even reports that Facebook decided against measures to reduce some of the harms (e.g., polarisation or mental health issues) because this would have harmed user engagement and put the growth of the company at risk \citep{seetharaman_facebook_2020, hao_how_2021}.

Facebook is even fairly open about this goal to maximise user engagement, mentioning it in several blog posts:
\begin{itemize}
    \item \citep{shen_making_2019} Facebook introduces ranking of comments also based on \textquote{[...] whether people like, react to, or reply to a comment.} (i.e., engagement).
    \item A report about expanding the \textit{Facebook News} service to more countries, states: \textquote{We will continue to focus on growing engagement of Facebook News [...]} \citet{brown_bringing_2020}.
    \item \citep{rubin_cloud_2020} reports about updates on Facebook cloud gaming (a service to attract gamers). Among others, they report about \textquote{updated discovery and re-engagement features}.
    \item A report about Facebook giving users more control over what they see also states that the news-feed will be influenced by: \textquote{Related Engagement} and what people recently engaged with  \citep{sethuraman_more_2021}.
\end{itemize}

Since Facebook earns most of its money trough advertisement, which depends on user engagement, there is reasonable doubt in how genuine Mark Zuckerberg's claim to focus on helping to find \textquote{meaningful social interactions} \citep{zuckerberg_one_2018} is.
As an article in \emph{The Guardian} in reaction to Zuckerberg's post pointed out, that this change in focus eventually benefits Facebook's growth.
First, personal posts are important to keep users engaged, that is, people tend to post more personal stories if their peers do so as well, and personal content is important for overall engagement.
A further side effect of this step could be the crowding out of news platforms, which in turn would create a more closed and self-contained Facebook, making the company ever more powerful.

\subsubsection{Science is complicated}
Subsection \ref{sub:contr} presented some points of criticism about research in the field, all of which are important to consider for future research in order to create a solid body of knowledge on problematic use of \gls{sns}, the internet, and smartphones.

However, such points of criticism, which are an essential part of any well functioning scientific community \citep{oreskes_why_2019}, should not distract us into concluding that science is completely uncertain about problematic use of \gls{sns}.
For this, let us contextualise the most prominent points of criticism.

\paragraph{Correlation vs. Causation} While it is true that correlational studies alone do not allow making causal inferences, they can still add valuable information to the body of knowledge.
Especially if there is a pre-existing body of research, such studies can be valuable first indicators of causal effects.\footnote{In the case of \gls{sns} addiction such mechanisms could be: The understanding of certain gratification mechanisms and their implications on the dopamine system together with knowledge about the influences of dopamine on depression.}
Or as Namoi Oreskes puts it in her Book \emph{Why Trust Science?}:
\begin{quote}
    But if we have an observed correlation between two phenomena, and we are aware of mechanisms that explain how one of them can be caused by the other, and if that mechanism is known to be present, then the logical conclusion is the observed correlation \emph{is} caused by the known mechanism. [...] Or at least, it is likely to be.
\end{quote}
\citep[p 116]{oreskes_why_2019}

\paragraph{Self reporting} Also finding no correlation of self reported measures with hard numbers of \gls{sns} usage can be interpreted in two (extreme) ways, and many shades in between:

First, all studies relying on self reporting did not measure anything meaningful related to digital media, thus they should be discarded.

Second, -- as elaborated above -- capturing all the facets of social media use in single numbers is difficult.
While self reporting might not directly relate to, for example, mere time spent on digital media, it could be a higher fidelity measure of the subjective experience during digital media use.
That is to say, self reports might not directly translate into hard numbers, however, they can still represent something meaningful.

\paragraph{Addiction vs. Problematic Use}
Much discussion circled around whether to call the problem an addiction or not.
There are good reasons to avoid using the term addiction excessively, and many people that would call themselves \gls{sns} addicts might clinically fall in the categories of overcompensation, or problematic use.
Indeed, this debate is important for members \textit{within} the scientific community, however, it tends to obfuscate the broader underlying consensus.
Namely, that \gls{sns} do -- for some users -- induce compulsive and excessive use and lead trough this to severe negative outcomes. 
While the definitions of what to call abnormal changed over time, there were always people falling into this category.

\subsubsection{We know the mechanisms}
Finally, even if non-representative correlational studies based on self reporting, provide only weak evidence,
we still know of mechanisms which lead to behavioural addictions.
For example random gratifications \citep{haw_random-ratio_2008}, endless scrolling and showing people what they like \citep{montag_addictive_2019}, or \textquote{message received} or \textquote{message read} notifications which tap into social feedback cycles \citet{stefanou_facebooks_2018}. 

In addition, we know that Facebook's services contain some of these mechanisms (some reports say deliberately \citep{manjoo_facebook_2017, seetharaman_facebook_2020, hao_how_2021}).
Therefore, we have good reasons to believe that there are addictive elements in \gls{sns} which may have severe negative effects for some users.

\subsection{Limitations}
The introduction to this case study, drew an analogy to Big Tobacco, which tried to selectively fund research to sow doubt and manufacture dissent. 
Moreover, a 2018 article by the \emph{New York Times} reports in-transparent media campaigns by Facebook trough \textit{Definers} a consulting firm that also ran \emph{NTK} -- an influential news-page -- among others to \textquote{muddy waters} \citep{frenkel_delay_2018}.
Facebook denies these accusations and specifically \textquote{[having] ever asked Definers to pay for or write articles on Facebook’s behalf – or to spread misinformation} \citep{facebook_new_2018}. 

Given the reports of such actions -- reminding of Big Tobacco -- it would certainly have been an interesting aspect to systematically assess influences of Big Tech on the scientific debate.
However, well grounded knowledge about Big Tobacco's practices is only available because of the extensive \textquote{documents unearthed by the tobacco legislation of the 1990s} \citep[p 297]{oreskes_merchants_2010}.
Furthermore, as we will see in section \ref{sec:fac_fund}, finding the funding sources of researchers is not always (easily) possible and often time-consuming. 
Therefore, this case study must be limited to present anecdotes of researchers industry ties.
Section \ref{sec:fac_fund} specifically, will present an anecdotal first attempt to understand Facebook's funding practices\footnote{It should be noted that the substantive scientific debates on \gls{sns} addiction do not indicate (e.g. trough obviously misdirected research) large scale attempts of influencing the field through funding., However, it would certainly be an interesting topic for future research to understand if Big Tech influenced scientific debates and if so, which debates?}.

Finally, I want to remark that I do not have any training in psychology or psychiatry, thus my elaborations on (behavioural)-addiction are limited to the status of outside observations. Furthermore, the unstructured approach to this case study cannot claim depicting a fully exhaustive picture of the debate.