\section{Conclusion}
Throughout this case study, some strategies outlined in the introduction reappeared in Facebook's demeanour.
The most prominent strategy is \emph{distraction}:
Facebook makes strong claims about the benefits of its service and argues, among others, that the service can be beneficial for depressed people, and if only using it the right way, everyone can benefit from Facebook.
At the same time, Facebook also uses the way science works (i.e., trough constant challenging of results, and a focus on uncertainties) to claim that there is no scientific consensus or that the science is not reliable.
While there is currently not enough evidence, to accuse Facebook of selling doubt, the company is certainly \emph{leveraging uncertainties} in the science on SNS addiction.

Additionally, freedom arguments are presented in the debate.
That is, people should never be restricted in their freedom to choose (if and) how they use Facebook.
Finally, while Facebook might not run \gls{pr} campaigns at the scale of Big Tobacco\footnote{One explanation for this might be that contrary to Big Tobacco, Facebook does not suffer declining users counts from current reports  \citep{oreskes_merchants_2010, tankovska_facebook_2021}.}, it still does create large campaigns to influence public opinion and more importantly it does spend large sums on lobbying\citep{chung_big_2021}\footnote{Figure \ref{fig:lob_exp} in appendix \ref{app:figures}, shows the development of lobbying expenditures by Facebook and Amazon in comparison with Exxon Mobile and Philip Morris over time.}.

\subsection{Implications for Law and Policy}
From a clinical perspective, talking about \gls{sns} or smartphone addiction does not seem to be useful, because of the danger of overpathologisation. 
The clinical diagnosis of an addiction implies difficult treatment and often social stigma. 

However, the clinical use of \textit{addiction} differs in meaning from the colloquial use of the term, such as saying \textquote{I am addicted to Instagram}. 
Such a personal report of addiction -- while not comparable  with substance addiction (e.g., cocaine) -- indicates an undesired condition, while people might not be able to report any specific, significant impairment during their everyday life, the downside can be no less real. 
Moreover, social acceptance and widespread use -- both of which allow heavy use without stigmatisation -- does not necessarily mean that \gls{sns} and smartphones are, in their current form, beneficial to society.
Therefore, I want to appeal to judiciaries and policymakers in that they:

\paragraph{Be alert} if industry representatives claim that science is not certain or unreliable. They might draw upon convincing dissenting opinions from renowned scholars published in prestigious journals. However, you should not forget that dissent is an integral part of science, and always ask if such an opinion has substantial implications for the underlying problem, or whether it is part of a more esoteric (i.e., directed towards the members of the community) scientific debate.

\paragraph{Focus on the mechanisms} Platforms like Facebook or Google are complex constructs that strive to unite more and more services into one self-contained and highly connected universe. 
This interrelated structure makes it virtually impossible to assess single features in isolation.
What is possible, however, is to understand certain recurring mechanisms (infinite scroll, random gratifications, etc.) which are built into these platforms. 
Thus, if possible, regulation should focus on underlying mechanisms.
This has the additional benefit of being less restricted to specific services -- which are ephemeral.

\paragraph{Treat cost-benefit arguments with caution} Cost-benefit analysis (CBA) is a powerful tool for policymakers, since it is sometimes able to reduce a complicated problem to a single dimension (i.e., Dollars or Euros).
However, this strength of CBA is also its limit.
Many problems simply can not be reduced to Dollars or Euros. 
Some might argue that it is the best tool we have, and you should not discard information. 
Yet, we also know that people are susceptible to anchoring\footnote{The Book by \citep{kahneman_thinking_2011}, is a good non-academic introduction to heuristics and biases.}, meaning you can be strongly influenced by numbers even those that have little in common with the problem they are trying to describe.

\paragraph{Invest into transparency of scientific funding} This recommendation is not restricted to the problem of \gls{sns} addiction, but is targeted at the bigger underlying issue. 
That is, the influence on seemingly independent science by large interest groups. 
Transparency is not valuable in and of itself.
However, easy access to information on potential conflicts of interest, can be a great asset when assessing the state of -- and consensus within -- a scientific field.

One potential way forward to more transparent scientific funding could be the creation of a platform where researchers can and are incentivized to make all their funding public. Such an incentive could be requiring the disclosure of all previous funding sources to be eligible for government funding.
Ideally, this would happen on a global scale.

\paragraph{Do not forget: Freedom is necessarily limited} An important and much-contested societal issue is freedom and the limitation of it. 
In many societies (esp. the US) freedom is an important value, and any kind of regulation is suspected to be an unnecessary restriction. 


Therefore, some argue that for regulation to be legitimate, there should be solid scientific evidence of harm. 
\todo{Fertig argumentieren}


\subsection{Limitations}
This case-study is certainly limited in that it has -- through Facebook as an object of interest -- a strong focus on the United States. 
Furthermore, some findings are specific to Facebook and do not necessarily generalize to other big technology companies (in this regard, the title might be slightly misleading). 
Finally, a substantial part of the arguments against Facebook builds on the accusations of a handful of former employees.
While there seem not to be strong reasons to disbelieve them, the knowledge in this case is not as solid as the case against Big Tobacco with an extensive collection of unearthed documents.

\subsection{Open Questions}
After writing this piece, two questions appear especially pressing:

First: Why did the SMART act fail so badly (It did not receive much support and died soon)? 
With its focus on mechanisms, the bill seemed (at least at first sight) to follow a sensible approach.
Thus, it would be interesting to understand which aspects were decisive for its failed progress: Was it the fear of being  paternalistic? Was it lobbying? or was the bill simply written badly?
These, and similar questions, should also be enlightening for the current attempts in the US to regulate Big Tech.\todo{Cite 1-2 newspapers}

Second: How did funding in the field of internet and \gls{sns} addiction develop over time?
As elaborated above, researching the affiliations of influential scientists in the internet and \gls{sns} addiction debate, was beyond the scope of this case study. 
Yet, this dimension would add interesting contextual information for understanding the debate. 
Such an attempt might best be aided by computational social sciences methods, for example the idea of scientific memes and their propagation \citep{kuhn_inheritance_2014}, to identify the introduction and development of concepts. 